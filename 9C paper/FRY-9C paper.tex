\documentclass[conference]{IEEEtran}
\IEEEoverridecommandlockouts
% The preceding line is only needed to identify funding in the first footnote. If that is unneeded, please comment it out.
%Template version as of 6/27/2024
\usepackage{amsmath,amssymb,amsfonts}
\usepackage{algorithmic}
\usepackage{graphicx}
\usepackage{textcomp}
\usepackage{xcolor}
\usepackage{hyperref}
\usepackage{url}
\usepackage{multirow}
\def\BibTeX{{\rm B\kern-.05em{\sc i\kern-.025em b}\kern-.08em
    T\kern-.1667em\lower.7ex\hbox{E}\kern-.125emX}}
\begin{document}

\title{US Bank Loan Portfolio Analytics Using Federal Reserve Regulatory Balance Sheet Filings: Methods, Trends, and Research Directions
}

\author{\IEEEauthorblockN{Venkat Chandrasekar Subramaniam}
\IEEEauthorblockA{\textit{DB Global Technology} \\
\textit{Deutsche Bank}\\
Cary, USA \\
venka\_t@yahoo.com}

}

\maketitle

\begin{abstract}
This document is a model and instructions for \LaTeX.
This and the IEEEtran.cls file define the components of your paper [title, text, heads, etc.]. *CRITICAL: Do Not Use Symbols, Special Characters, Footnotes, 
or Math in Paper Title or Abstract.
\end{abstract}

\begin{IEEEkeywords}
component, formatting, style, styling, insert.
\end{IEEEkeywords}

\section{Introduction}
As part of its macroprudential and microprudential supervision process mandated post the 2008 Global Financial Crisis, the Federal Reserve requires the Bank Holding Companies(BHCs), Savings and Loan Holding Companies(SLHCs) and Intermediate Holding Companies(IHCs) to file various reports. These reports may be on a daily, monthly , quarterly , annual or on in an as required basis. 

All BHCs, SLHCs and IHCs(called Banks hereon) are required to file the FR Y-9C report on a quarterly basis by banks having assets more than \$ 50 Billion or more.This information is used by the Federal Reserve(Fed) to monitor the health of the banks in between inspections.\cite{Fed9C}

In this paper , we will discuss on the opportunities for analytics given by the vast data provided in the FRY-9C.
\subsection{Role of BHCs in US Credit Market}
     BHCs by virtue of their size provide the much required stability to the banking system. BHCs provide approximately 63\% of the total credit in the US as of 2018 which provide the much needed liquidity to the Credit markets. Also since BHCs are subject to enhanced oversight and so any minor glitches in the performance can attract the attention of regulatory agencies like the Fed.
     
     BHCs also offer the much required diversification since the traditional interest income sources do not offer the required economies of scale for sustainability of banks the size of the BHCs and so BHCs are constantly on the lookout for opportunities to provide risk capital. This activity makes BHCs inherently innovative enabling BHCs to diversify their lending. BHCs lend to diversified obligors like private creditors and also engage in market making and underwriting activities.
     
     All the above make the BHCs systemically important despite their critics saying that the large banks(which were the forefathers of the BHCs) were the cause of the 2008 meltdown.
     
\subsection{Reason for considering the FR Y-9C data for Bank analysis}
       The FR Y-9C provides a rich set of data which is available in public in the Federal Financial Institutions Examination Council(FFIEC) website. While the data is provided on an aggregate basis the data has been provided in such a way as to lend itself to further analysis. Further the FR Y-9C is a report which has been specifically designed for Banks and while banks form 63 \% of the total credit only, they form the bulk of the data required for the analysis for the purpose of this paper. Below, I present some of the reasons choosing FR Y-9C over reports like SEC 10-Q Report or NCUA Call reports.
\subsection{The FRY-9C Vs the SEC 10-Q}

The SEC 10-Q is a quarterly financial report which should be filed by publicly traded companies in the United States.\cite{SEC10Q} The SEC 10-Q need not be filed by privately held companies. While all banks are publicly traded in the US due to the capital requirements post the 2008 Global Financial Crisis, the SEC-10Q has got limitations in the analysis of banks.The SEC 10-Q report is a freeform report in that it does not have a specific structure. Because it does not have a specific structure unlike most of the Fed Reports which have a specific structure with instructions for each data item or MDRM reported.\cite{MDRM}

The SEC 10-Q is also a report which is more generically tailored for generic company financials, which may include banks as well as non banks. Given that this paper is about bank loans, the SEC 10-Q dataset would be too comprehensive for the purposes of this paper and using the SEC 10-Q data will inject considerable extra effort in the process. 

\subsection{The FRY-9C vs the NCUA Call Reports}
The National Credit Union Administration(NCUA) requires all Credit Unions to file comprehensive reports at the level of each loan after is available for public consumption its website. These data dumps provide a wealth of information concerning the invdividual loans lent by each credit union.\cite{NCUA} However since we are interested in the performance of bank loans on an aggregation, we will ignore this dataset for the purposes of this paper. The figure below shows a comparison of the approximate filers for FR Y-9C, SEC 10-Q and the NCUA Call reports. Data taken from \cite{9CCount}, \cite{SECCount},\cite{NCUACount}
\begin{figure}[htbp]
	\centerline{\includegraphics{9C Vs 10-Q Vs NCUA Grayscale.png}}
	\caption{Filer count FR Y-9C, SEC 10-Q, NCUA Call Reports}
	\label{fig}
\end{figure}

\section{History of the FR Y-9C}
Before I delve into the history of FRY-9C, the concept of a BHC must be explained.
\subsection{The BHC Concept}
 The concept of a bank holding company(BHC) came into existence during the mid-1920s with the Fed proposing it in 1927 . The Glass Steagall Act of 1933 provided for the separation of the banking and non banking activities adding more teeth to the principle of BHC. Changes to the construct were made in 1970 when the BHC act was amended. The Dodd Frank Act following the Global Financial Crisis of 2008 created more restictions on BHC while bringing in more banks under the cover of a BHC owing to the ease of supervision under the structure of a BHC.\cite{BHC}

\subsection{The FR Y-9C History}
    According to the Fed in its website, the report started as FRY-9 in 1978. In 1985 report was changed to act as a parallel report to the call reports which are filed in various capacities by banks. For example Deposit taking banks may based on whether they have foreign offices or not need to file the reports of FFIEC-031 or FFIEC-041 which are called the call reports. 
    
    In 1986, the FR Y-9 was split into FR Y-9C (consolidated statement) and FR Y-9LP(Parent company only financial statements) . The threshold for filing the report was changed from \$ 150 million to \$ 3 billion between 2006 and 2018 in various stages. 
    
    The FR Y-9C is required to be filed by BHCs on a quarterly basis in accordance with regulation Y. In 2011, the Dodd Frank Act abolished the Office of Thrift Supervision and SLHCs , except for exempt SLHCs were required to file the FR Y-9C. There was a 2 year phase in period for SLHCs starting Q1 2012 for starting to file FR Y-9C. As per Regulation YY , IHCs of Foreign Banks were also required to file FR Y-9C from 2016.\cite{Fed9C}. The data is summarized as table below
    
    \begin{table}[htbp]
    	\centering
    	\caption{Timeline of FR Y-9C Reporting Requirements\cite{Fed9C}}
    	\begin{tabular}{|p{1.5cm}|p{6cm}|}
    		\hline
    		\textbf{Year} & \textbf{Event} \\
    		\hline
    		1978 & Introduction of FR Y-9 report. \\
    		\hline
    		1985 & Modified to act as parallel to bank Call Reports (FFIEC 031/041). \\
    		\hline
    		1986 & Split into FR Y-9C (Consolidated) and FR Y-9LP (Parent only). \\
    		\hline
    		2006--2018 & Filing threshold raised from \$150M to \$3B. \\
    		\hline
    		2011 & Dodd-Frank: OTS abolished; SLHCs (except exempt) required to file. \\
    		\hline
    		2012--2014 & Two-year phase-in for SLHCs starting Q1 2012. \\
    		\hline
    		2016 & Reg. YY: IHCs of foreign banks required to file. \\
    		\hline
    	\end{tabular}
    \end{table}
\section{FR Y-9C for Bank Loan Analysis}
      The FR Y-9C report is apt for data analysis for the following reasons. The FR Y-9C has got a very well defined MDRM structure where there are rules provided in very minute detail for each MDRM in the report. This makes the report consistent and the reliability of data provided by each bank in each of the MDRM is high. In order to maintain consistency of the data, the Fed publishes a set of Edit Checks as well with the FR Y-9C instructions. A section of these Edit Checks must be satisified in order for banks to even submit the report.These checks and consistent clean data reduce a lot of time spent by researchers in processing the data for their analysis. 
      
      Also the FR Y-9C report data is available for public consumption in the FFIEC website and so considerable time is reduced in obtaining the required permissions for accessing confidential data. This makes data availability faster.
      
      All the above combined with the rich dataset of around 2000 data points or MDRMs provided in FR Y-9C makes it an ideal candidate for Bank data analysis and in specific BHC Loan data analysis.
\section{Purpose of this paper}
This paper attempts to provide a review of the data available in the FR Y-9C report. If the facts provided previously in this paper are taken into consideration and a search is done for scholastic material referencing the FR Y-9C alone, the amount of papers obtained are limited. Given the huge opportunity which lies untapped, this paper attempts to increase the consumption of data from the FR Y-9C for the purposes of research into the behavior of banks and for the economic analyses possible using this very rich dataset provided for free.

Possible areas where FR Y-9C data can be leveraged include usage of FR Y-9C MDRMs as independent variables in the regression equations . Xiangchao et al. \cite{SecRef} use the FR Y-9C data in their investigation to check whether Securitization and CDS have an effect on US Bank Lending . For testing their hypothesis they use the FR Y-9C, they use data from HC-C(Loans and Lease Financing Receivables), HC-R(Risk Weighted Assets) and HC-S(Securitization) to check if CDS and Securitization affect loans growth.

Abdul-Khalik and P.C. Chen use the derivatives data from FRY-9C to examine the impact of derivative trades before and after FAS133 and the use of derivatives by banks to hedge risk in their paper . \cite{der9C}


\section{Brief description of the schedules of FRY-9C}
    The FR Y-9C has got 24 schedules with a number of data points called MDRMs within each schedule. The description of each of the schedules are listed below in the table.
    \begin{table}[htbp]
    	\centering
    	\caption{FR Y-9C Schedules\cite{Fed9C}}
    	\begin{tabular}{|p{1.5cm}|p{6cm}|}
    		\hline
    		\textbf{Schedule} & \textbf{Description} \\
    		\hline
    		HI & Consolidated Income Statement \\
    		\hline
    		HI-A & Changes in Equity Capital \\
    		\hline
    		HI-B & Charge-Offs and Recoveries on Loans and Leases and Changes in Allowances for Credit Losses \\
    		\hline
    		HI-C & Disaggregated Data on the Allowance for Credit Losses \\
    		\hline
    		ISnotes-P & Notes to the Income Statement — Predecessor Financial Items \\
    		\hline
    		ISnotes & Notes to the Income Statement — Other \\
    		\hline
    		HC & Consolidated Balance Sheet \\
    		\hline
    		HC-B & Securities \\
    		\hline
    		HC-C & Loans and Lease Financing Receivables \\
    		\hline
    		HC-D & Trading Assets and Liabilities \\
    		\hline
    		HC-E & Deposit Liabilities \\
    		\hline
    		HC-F & Other Assets \\
    		\hline
    		HC-G & Other Liabilities \\
    		\hline
    		HC-H & Interest Sensitivity \\
    		\hline
    		HC-I & Insurance-Related Underwriting Activities (Including Reinsurance) \\
    		\hline
    		HC-K & Quarterly Averages \\
    		\hline
    		HC-L & Derivatives and Off-Balance Sheet Items \\
    		\hline
    		HC-M & Memoranda \\
    		\hline
    		HC-N & Past Due and Nonaccrual Loans, Leases, and Other Assets \\
    		\hline
    		HC-P & Closed-End 1-4 Family Residential Mortgage Banking Activities \\
    		\hline
    		HC-Q & Financial Assets and Liabilities Measured at Fair Value \\
    		\hline
    		HC-R & Regulatory Capital \\
    		\hline
    		HC-S & Servicing, Securitization, and Asset Sale Activities \\
    		\hline
    		HC-V & Variable Interest Entities \\
    		\hline
    	\end{tabular}
    \end{table}
    
    
    While each schedule might directly or indirectly contribute to the overall loan data the most important schedules which deal with loan data are HC-C Loan and Lease Financing Receivables, HC-N Non Accrual Loans and HC-L Derivatives and Off Balance Sheet items. 
    
    \subsection{HC-C Loans and Lease Financing receivables}
     This section which falls under HC gives details about the Loans and Lease receivables by the bank which the bank intends to hold in its books. Lines 1 to 10 talk about loans post which Lease details kick in. The descriptions of each of the Loan lines is given in Table VII under Appendix.The loans are organized into three sections each section based on Collateral, Borrower and Purpose.
     
     The entire schedule is divided into two columns with one column for the data at a consolidated level and the other for data specifically reported for the domestic offices. Usually MDRMs fall into one or the other category though there are some lines having MDRMs for both categories.
     
     With regard to the individual categories of Collateral, Borrower or Purpose, each MDRM usually falls under one category though the same MDRM falling under two or more categories is also possible. 
     
     \subsubsection{Collateral}
            The MDRMs under this category have the collateral as one of the following
             \begin{table}[htbp]
            	\centering
            	\caption{Collateral Types in HC-C}
            	\begin{tabular}{|p{1.5cm}|p{6cm}|}
            		\hline
            		\textbf{Collateral Type} & \textbf{Lines which fall under category} \\
            		\hline
            		Real Estate & 1 Col A, 1.a.1 Col B, 1.a.2 Col B,1.b,1.c.1,1.c.2.a,1.c.2.b,1.d,1.e.1,1.e.2 \\
            		\hline
            		Unsecured Lending & 6.a,6.b \\
            		\hline
            		Automobile & 6.c \\
            		\hline
            	\end{tabular}
            \end{table}
      
      \subsubsection{Borrower}
            This category has a slight overlap with the type of collateral. Lines under this category are
            \begin{table}[htbp]
            	\centering
            	\caption{Kinds of borrowers with data in HC-C}
            	\begin{tabular}{|p{3.5cm}|p{4cm}|}
            		\hline
            		\textbf{Borrower Type} & \textbf{Lines which fall under category} \\
            		\hline
            		Type of occupier & 1.e.1 ,1.e.2 \\
            		\hline
            		Type of banks & 2.a,2.b \\
            		\hline
            		Farmers & 3 \\
            		\hline
            		Based on address & 4.a,4.b\\
            		\hline
            		Government institutions&7\\
            		\hline
            		Non depository institutions&9.a,9.b.1,9.b.2,9.b.3\\
            		\hline
            		\end{tabular}
            \end{table}
        \subsubsection{Purpose}
             Like the other categories the category of purpose also is split into multiple categories with the type of borrower overlapping with the purpose. Below gives the specific purpose of each of the purpose based detail captured in FR Y-9C
             \begin{table}[htbp]
             	\centering
             	\caption{Purpose Segregation in HC-C}
             	\begin{tabular}{|p{3.5cm}|p{4cm}|}
             		\hline
             		\textbf{Purpose} & \textbf{Lines which fall under category} \\
             		\hline
             		Agriculture & 3. \\
             		\hline
             		Commercial and Industrial Loans & 4.a,4.b \\
             		\hline
             		Personal Expenditure & 6 \\
             		\hline
             		Securities and Financial Transactions & 9.b.1,9.b.3\\
             		\hline
             	\end{tabular}
             \end{table}
             
             \subsubsection{HC-C Memo}
             	HC-C has got a memorandum section as well. This memoranda section deals with special loans conditions in HC-C . The overall theme for each line is given in the \ref{HCCMemo}
             \begin{table}[htbp]
             	\centering
             	\caption{Schedule HC-C Memoranda Items Summary (with Subordinate Lines)}
             	\label{HCCMemo}
             	\begin{tabular}{|p{0.5cm}|p{3cm}|p{2.5cm}|p{1cm}|}
             		\hline
             		\textbf{Line} & \textbf{Description} & \textbf{Commentary} & \textbf{Sub-Lines} \\
             		\hline
             		M.1 & Loan modifications to borrowers experiencing financial difficulty & Loan modifications so reported in HC-N Memo & M.1.a– M.1.e \\
             		\hline
             		M.2 & CRE construction loans not secured by real estate & Unsecured Real estate construction loans & -- \\
             		\hline
             		M.3 & Loans to non-U.S. addressees & Foreign borrower exposure & M.3.a– M.3.e \\
             		\hline
             		M.4 & Credit card fees and finance charges & Credit card Fees and charges & -- \\
             		\hline
             		M.6 & Negative amortization loans & Loans with negative amortization features grouped by product & M.6.a– M.6.c \\
             		\hline
             		M.9 & Foreclosure activity & Loans which are in the process of Foreclosure during end of reporting period. & -- \\
             		\hline
             		M.12 & Business combination acquisitions & Loans in book due to M\&A  & M.12.a– M.12.c \\
             		\hline
             		M.14 & Pledged loans & Loans pledged as collateral & -- \\
             		\hline
             		M.15 & Converted home equity lines & HELOCs converted to amortizing loans & -- \\
             		\hline
             	\end{tabular}
             \end{table}
             
            		
\subsection{HC-N Past Due and Non Accrual Loans and Leases}
	Schedule HC-N of FR Y-9C provides details about the quality of the loan portfolio of the bank in terms of debt repayments by the counterparties for each loan. This data is provided at an aggregated loan basis. In order to have a consistent view of the performance of each segment of the loan portfolio, HC-C and HC-N have the same row taxonomy till Line 4. 
	
	However each MDRM in HC-C corresponds to three MDRMs in HC-N with the exception of the domestic offices MDRM for HC-C Line 3 which has been totally ignored for the purposes of HC-N. The reason for 3 MDRMs is because each line item in HC-C has been split based on the days past due for repayments for each of the loans. There are three buckets defined for days past due. The first bucket is days past due from 30 to 89 and the second bucket is the days past due from 90 to 180 but still the conditions of non accrual are not met. These two buckets have the loan interest still accruing. The condition for non accrual are threefold :
	
	1. The underlying asset is liquidated and so the loan is at a cash basis 
	2. Payment of Principal and interest due in full is not expected
	3. Principal and/or interest is past due for more than 90 days
	
	Please note that loans which dont have a past due status need not be reported in HC-N. The reporting data representation in HC-N is given in \eqref{HCNFig}
	
	\begin{figure}[htbp]
		\centerline{\includegraphics{hcn_flowchart.png}}
		\caption{HC-N Buckets}
		\label{HCNFig}
	\end{figure}
	
	There are also certain shifts between HC-C and HC-N with regard to line numbers but there will be a relationship between an MDRM in HC-C and HC-N. The Fed does not remove lines when MDRMs are not required for reporting but makes the particular line not applicable to maintain consistency while referring line numbers in the reports. So these redundant MDRMs are removed and the corresponding line numbers are marked non-applicable. Removing the lines in such cases, the nature of the lines in HC-C and HC-N usually similar till Line 8 HC-C. Line 7 in HC-N is the catch all bucket for loans in each status which is unlike Line 9 in HC-C which details loans to non-depository financial institutions.
	
	\subsubsection{HC-N Memo}
		Similar to HC-C , HC-N also has a memo section which contains a lot of useful data for analysis of a loan of a BHC. The line structure of the HC-N Memo for lines from line 1 with its subordinate lines through Line 3 follow the same description as HC-C Memo except that in HC-C the overall amounts are reported and in HC-N Memo the loans in verious stages of non-accrual as provided. 
		
		HC-C Memo line 4 has no equivalent in HC-N Memo since it is assumed that Credit Card Fees comes into play mostly when the dues on Credit Card start getting past due which has already been captured in HC-C Memo.
		
		HC-N Memo Line 5 deals with Loans which are in various stages of non-accrual which are held for sale, probably for loans to be sold as credit deteriorated loans. Line 6 deals with Derivative contracts created on top of the loans in only the two past due states of 30-89 days and 90 days or more which are still not in non-accrual state. Lines 7 and 8 deal with the amount of non-accrual assets which are present in the balance sheet of the bank by tracking the additions and sales of non-accrual assets. Lines 7 and 8 are very important to track the health of the loan portfolio of the bank which is either held for sale(HFS) or held for investment(HFI). The mappings between HC-C Memo and HC-N Memo for lines 1 to 3 is given in the table \eqref{tab:hcn_hcc_memo_mapping}. As mentioned earlier there is a one on one mapping between HC-C Memo and HC-N Memo for the first 3 Memo lines.
		
		\begin{table}[htbp]
			\centering
			\caption{Mapping of Schedule HC-N Memoranda to Schedule HC-C Items}
			\label{tab:hcn_hcc_memo_mapping}
			\begin{tabular}{|p{1.5cm}|p{3cm}|p{2.5cm}|}
				\hline
				\textbf{HC-N Memo Line} & \textbf{HC-N Memo Description} & \textbf{Corresponding HC-C Line} \\
				\hline
				M.1.a.(1) & Construction, land development, and other land loans: 1-4 family residential construction loans & HC-C, M.1.a.(1) \\
				\hline
				M.1.a.(2) & Construction, land development, and other land loans: Other construction loans and all land development and other land loans & HC-C, M.1.a.(2) \\
				\hline
				M.1.b & Loans secured by 1-4 family residential properties in domestic offices & HC-C, M.1.b \\
				\hline
				M.1.c & Secured by multifamily (5 or more) residential properties in domestic offices & HC-C, M.1.c \\
				\hline
				M.1.d.(1) & Secured by nonfarm nonresidential properties: Loans secured by owner-occupied nonfarm nonresidential properties & HC-C, M.1.d.(1) \\
				\hline
				M.1.d.(2) & Secured by nonfarm nonresidential properties: Loans secured by other nonfarm nonresidential properties & HC-C, M.1.d.(2) \\
				\hline
				M.1.e.(1) & Commercial and industrial loans: To U.S. addressees (domicile) & HC-C, M.1.e.(1) \\
				\hline
				M.1.e.(2) & Commercial and industrial loans: To non-U.S. addressees (domicile) & HC-C, M.1.e.(2) \\
				\hline
				M.1.e.(3) & Commercial and industrial loans: To U.S. addressees (domicile) and non-U.S. addressees (domicile) & HC-C, M.1.e.(3) \\
				\hline
				M.1.f.(1) & All other loans: Loans secured by farmland in domestic offices & HC-C, M.1.f.(1) \\
				\hline
				M.1.f.(2) & All other loans: Loans to finance agricultural production and other loans to farmers & HC-C, M.1.f.(2) \\
				\hline
				M.1.f.(3)(a) & All other loans: Credit cards & HC-C, M.1.f.(3)(a) \\
				\hline
				M.1.f.(3)(b) & All other loans: Automobile loans & HC-C, M.1.f.(3)(b) \\
				\hline
				M.1.f.(3)(c) & All other loans: Other consumer loans & HC-C, M.1.f.(3)(c) \\
				\hline
				M.1.f & All other loans (include loans to individuals for household, family, and other personal expenditures) & HC-C, M.1.f \\
				\hline
				M.1.g & Total loan modifications to borrowers experiencing financial difficulty included in Schedule HC-N items 1 through 7 & HC-C, M.1.g \\
				\hline
				M.2 & Loans to finance commercial real estate, construction, and land development activities (not secured by real estate) & HC-C, M.2 \\
				\hline
				M.3 & Loans and leases included in Schedule HC-N, items 1, 2, 4, 5, 6, 7, and 8 extended to non-U.S. addressees & HC-C, M.3 \\
				\hline
			\end{tabular}
		\end{table}
\subsection{HC-L Derivatives and off balance sheet items}
    Schedule HC-L of FR Y-9C deals with derivatives and off balance sheet items. While derivatives need not be directly linked to a loan, derivatives can be used for hedging the risks of loans. In this angle, the most important aspects of derivatives covered in HC-L are unused commitments,Letters of Credit and guarantees. The previously mentioned instruments may be provided at a counterparty level and sometimes not exactly explicitly at a loan level. However these can be used as handy means of analytics. Given that all the three mentioned are off balance sheet items , taking these data from FR Y-9C makes the FR Y-9C a very handy for an overall analysis of Off balance sheet items.
    
    The use of these off balance sheet items may be varied. Firms and individuals who have got facilities which are not unconditionally cancellable tend to use these facilities in a higher manner during times of stress similar to individuals using their credit cards higher during times of need.
    
    A letter of credit is an implicit guarantee by the bank stating to pay in case the borrower does not pay. These letters of credit are common in trade finance and again are much sought after instruments spiking in use during economic downturns when trust on the counterparties is at a low.
    
    HC-L data is also used in the CCAR(Comprehensive Capital Analysis and Review) or DFAST(Dodd Frank Annual Stress Test) modeling to model the growth of credit derivatives and other risk mitigants include the use of these instruments during periods of stress. A useful corollary on the use of these contingent instruments is that bank liquidity takes a beating and so we can use liquidity and use of the contingent instruments as a barometer to measure a downturn.
    
    The fees generated out of these contingent instruments form a major part of the non interest income for any bank and so analysis of these commitments provides a better view of the income statements of banks as a whole.
    
    \subsubsection{Commitments}
         For the purpose of loan analysis , a commitment refers to a pledge made by the bank to a counterparty to fund a particular amount. Based on the loan contract terms a commitment may be conditionally or unconditionally cancellable.  
         
         In the FR Y-9C the lines under line 1 are the lines to be reported under unused commitments. These commitment data can be loosely linked to the HC-C loans to get a better picture of the loan book of the bank. The table below gives an idea of the linkages.
         \begin{table}[htbp]
         	\centering
         	\caption{Mapping of HC-L to Equivalent HC-C Lines}
         	\begin{tabular}{|p{1.5cm}|p{1.5cm}|p{4cm}|}
         		\hline
         		\textbf{HC-L Line} & \textbf{Equivalent HC-C Line} & \textbf{Remark} \\
         		\hline
         		1.a & 1.c.1 &  \\
         		\hline
         		1.b.1 & 6.a &  \\
         		\hline
         		1.b.2 & 6.b &  \\
         		\hline
         		1.c.1.a & 1.a.1 &  \\
         		\hline
         		1.c.1.b, 1.c.2 & 1.a.2 & Commercial real estate is included in HC-L but not in equivalent HC-C Line \\
         		\hline
         		1.d & No equivalent in HC-C &  \\
         		\hline
         		1.e.1 & 4.a, 4.b, 4.c &  \\
         		\hline
         		1.e.2 & 2.a, 2.b, 9.a, 9.b.1, 9.b.2, 9.b.3 &  \\
         		\hline
         		1.e.3 &  & Catch all bucket for all other commitments \\
         		\hline
         	\end{tabular}
         \end{table}
         
         Since the schedule HC-L of FR Y-9C deals with Off Balance sheet items alone, only the unused portion of the commitments need to be reported.
       
       \subsubsection{Letters of Credit}
       Banks are required to report three types of Letters of Credit which are 
       
       1. \textbf{Financial standby letters of credit} : These are letters of credit which are given to guarantee performance of financial instruments issued by the client. These are irrevocable and long term. These letters of credit are reported in line 2.a of schedule HC-L
       
       2. \textbf{Performance standby letters of credit} : These are guarantees which are provided on a contractual basis for the bank's clients for specific contracts where the bank has to pay in case the performance of the client in certain aspects of the contract are not met. These are reported in lines 3 and 3.a
       
       3. \textbf{Commercial letters of credit} : These are guarantees for trade finance contracts issued on behalf of client for certain trades and have a smaller scope than performance letters of credit. These are reported in line 4 of HC-L.
       
       \subsubsection{Credit Derivatives}
       Credit Derivatives are reported in two sets of lines in Schedule HC-L. These derivatives are usually at a counterparty level though it can be taken at a transaction level as well. Lines 7.a.1 through 7.a.4 report the Notional amounts for Credit Default Swaps(CDS), Total Return Swaps(TRS), Credit Options and other exotic derivatives. While Total Return Swaps are not exactly used in loans, there is a possibility of clients dealing with banks using Total Return Swaps to provide temporary relief and so are taken into the fold of loans. 
       
       Given the sensitivity of these Credit Derivatives, multiple aspects of the derivatives are analyzed. 7.a looks at the notional amounts, 7.b looks at the Gross fair value, 7.c looks at the notional amounts by regulatory treatment and 7.d looks at the remaining maturity. In addition the same set of credit derivatives are analyzed by the type of contracts i.e Interest Rate Contracts, Equity Contracts, Foreign Exchange Contracts and Commodity Contracts in lines 11 through 14 in HC-L.
       
       The level of detail gone into for Credit Derivatives provides huge opportunities for analytics. These Credit derivatives may not be directly involved in Loan analytics but can be used in counterparty analytics as required.
    
    \subsection{HI-B Charge offs and Recoveries in Loans and Leases and Allowances for Credit Losses}
    Schedule HI-B has got two sections . The first section deals with the charge offs and recoveries. Column A is for the charge off and column B is for the recovery. The row structure followed in part I is similar to HC-N. 
    
    Part II is for the changes in allowances for credit losses. The actual amount of allowance provided will be available in Schedule HI-C \ref{HI-C}.  An allowance is the amount reserve the bank has kept for credit losses and provisions are the amounts added or reduced from the allowance during quarterly intervals. This has three columns which are Loans and leases held for investment, loans and leases held to maturity and loans and leases available for sale.
    
    \subsubsection{HI-B Memo}
    	HI-B has also got a Memo section . While this section is not detailed, it provides important information for analysis of the loan book of the BHC. Memo Line 1 deals with the additions to the Transfer Risk Reserve \cite{FDICRisk}. The Transfer Risk Reserve is a reserve against the possibility that the borrower outside of U.S might have difficulty in getting the home currency converted to US Dollar. In other words, transfer risk can be thought of as a country specific risk. The amount of reserves to be held as a percentage of the amount of loans provided by the Intra-agency Country exposure review committee(ICERC) under the FFIEC.
    	
    	Memo Lines 2 and 3 refer to the allowances for Credit Card losses which are either collectible or deemed uncollectible. In memo lines 4 , 5 and 6 the allowances  on credit losses are to be reported. Line 4 is for the Provision for credit losses for assets measured at amortized cost and 5 is the allowance maintained for the same set of assets. These assets are mostly Held to Maturity(HTM) assets (refer table \eqref{PFtype}). Line 7 is for the allowance for Off Balance sheet trades. Finally Memo Line 8 requires the estimated recoveries of amounts written off for credit losses to be reported.
    	
    \subsection{HI-C Disaggregated data on Allowance for Credit Losses}
    \label{HI-C}
    	This section provides the balances available in the Allowances available for credit losses. Due to addition and consumption of provisions for credit losses, the allowance for credit losses might. While HI-B part II provides for the changes in the allowances due to provisions, the amount of balances available in the allowances are detailed in HI-C in a disaggregated fashion.
    
    	The mapping from HI-C to HC-C is given in Table IX \eqref{tab:hic_hcc_mapping}. While data is more granular in HC-C when compared to HI-C, the comparison gives the required allowance to loan mapping on an aggregate level from a FR Y-9C perspective. HI-B Memo Line 8 can also be thought of as the amount recovered from the allowances which are disaggregated in Schedule HI-C.
    	
    	\begin{table}[htbp]
    		\centering
    		\caption{Mapping of Schedule HI-C (Allowances for Credit Losses) to Schedule HC-C (Loans and Leases)}
    		\label{tab:hic_hcc_mapping}
    		\begin{tabular}{|p{1.5cm}|p{4cm}|p{1.5cm}|}
    			\hline
    			\textbf{HI-C Line} & \textbf{HI-C Description (Loans and Leases Held for Investment)} & \textbf{Correspon- ding HC-C Line(s)} \\
    			\hline
    			Item 1.a & Construction loans & HC-C, 1.a.(1)--1.a.(2) \\
    			\hline
    			Item 1.b & Commercial real estate loans & HC-C, 1.b,  1.e.(1)--1.e.(2) \\
    			\hline
    			Item 1.c & Residential real estate loans & HC-C, 1.c.(1), 1.c.(2)(a)--1.c.(2)(b),1.d \\
    			\hline
    			Item 2 & Commercial loans & HC-C, 2.a--2.b, 3, 4, 7, 9.a--9.b.(2), \\
    			\hline
    			Item 3 & Credit cards & HC-C, 6.a \\
    			\hline
    			Item 4 & Other consumer loans & HC-C, 6.b, 6.c, 6.d \\
    			\hline
    			Item 5 & Unallocated, if any & No direct mapping possible to HC-C Lines  \\
    			\hline
    		\end{tabular}
    	\end{table}
    	
    \subsection{HC-D Trading Assets and Liabilities}
	While loans dont form a primary part of traded assets, the portion of the banking portfolio which is part of the trading book is captured in Line 6 in Schedule HC-D of FR Y-9C. The subordinate lines for these lines include Loans secured by 1-4 family, other real estate, commercial and industrial loans, personal loans and other loans. The loans provided in these lines are only those loans which are in the held for trading(HFT) category for the bank. A comparison is given in Table \eqref{PFtype} for reporting of each category of the bank portfolio which includes the Available for sale(AFS) portfolio as well.
	
	\begin{table}[htbp]
		\centering
		\caption{Comparison of HFT, AFS, and HTM Portfolios in FR~Y-9C Reporting}
		\label{PFtype}
		\begin{tabular}{|p{1.2cm}|p{2cm}|p{2cm}|p{2cm}|}
			\hline
			\textbf{Feature} & \textbf{Held for Trading (HFT)} & \textbf{Available-for-Sale (AFS)} & \textbf{Held-to-Maturity (HTM)} \\
			\hline
			\textbf{Intent} & {Short-term resale or market-making} & {Manage liquidity or ALM; may sell if needed} & {Hold to maturity, collect contractual cash flows} \\
			\hline
			\textbf{FR~Y-9C Schedule} & {HC-D (Trading Assets)} & {HC-B (Securities)} & {HC-B (Securities)} \\
			\hline
			\textbf{Valuation} & {Fair value} & {Fair value} & {Amortized cost} \\
			\hline
			\textbf{Unrealized Gains/Losses} & {Through net income (HI)} & {Other comprehensive income (OCI)} & {Not recognized (unless impaired)} \\
			\hline
			\textbf{Typical Assets} & {Trading loans, derivatives, securities} & {Bonds, MBS, munis} & {Treasuries, agency securities} \\
			\hline
			\textbf{Risk Focus} & {Market \& credit spread risk} & {Interest rate \& liquidity risk} & {Credit risk, duration stability} \\
			\hline
		\end{tabular}
	\end{table}
	
	
	The loans reported in this schedule are valued at fair value unlike in HC-C where they are valued at amortized cost. Also as a corollary to the previous statement, the loans reported in HC-C and HC-D are mutually exclusive.
	
	\subsubsection{HC-D Memo}
		In HC-D Memo, Memo Lines 1 and 2 deal with the HFT loans which are reported in HC-D. In Memo Line 1 the unpaid principal balances for each of the loans reported in HC-D is to be reported. The reporting is to be disaggregated by whether the loan is secured by real estate, is a commercial loan,or if it is loans provided to an individual or any other loans. Line 2 collects data on the set of loans which have fallen into the non-accrual bucket and where the last payment is pending more than 90 days. Memo Line 2 requires the fair value and the unpaid principal balance of such loans to be reported.
		
	\subsection{HC-K Quarterly Averages}
	In addition to point in time data reported at the end of the quarter, the Fed also requires BHCs to report the quarterly averages. This data helps understand the quarterly pattern and if there was any last minute window dressing that happened towards the end of the quarter. Loans are represented in line 3 of HC-K. Given that the focus is on quarterly averages rather than the details of the loan, HC-K is not as comprehensive as HC-C in handling granular loan data as can be in the table below comparing the HC-K quarterly average lines vs HC-C lines.
	
\begin{table}[htbp]
	\centering
	\caption{Mapping of HC-K Line 3 (Average Total Loans and Leases) to HC-C Categories}
	\begin{tabular}{|p{2.2cm}|p{1.5cm}|p{4cm}|}
		\hline
		\textbf{HC-K Line} & \textbf{Equivalent HC-C Categories} & \textbf{Description} \\
		\hline
		\multirow{4}{*}{\parbox{2.2cm}{HC-K, Line 3.a.1: Average total loans and leases, net of unearned income, secured by 1-4 family residential properties}} 
		& HC-C, 1.c.1 & Loans secured by 1-4 residential mortgages, open ended revolving \\
		\cline{2-3}
		& HC-C, 1.c.2.a & Loans secured by 1-4 residential mortgages, closed, secured by first liens \\
		\cline{2-3}
		& HC-C, 1.c.2.b & Loans secured by 1-4 residential mortgages, closed, secured by junior liens \\
		\hline
		\parbox{2.2cm}{HC-K, Line 3.a.2: Average total loans and leases, all other loans secured by real estate}
		& HC-C, 1.a.1--1.a.2 & Loans secured by construction, development and other land loans \\
		\hline
		\parbox{2.2cm}{HC-K, Line 3.a.3: Average total loans and leases, loans to finance agricultural production}
		& HC-C, Line 3 & Loans to finance agricultural production and other loans to farmers \\
		\hline
		\parbox{2.2cm}{HC-K, Line 3.a.4: Average total loans and leases, commercial and industrial loans}
		& HC-C, Line 4.a and 4.b & Commercial and industrial loans to US and non-US addresses \\
		\hline
		\parbox{2.2cm}{HC-K, Line 3.a.5.a--3.a.5.b: Loans to individuals for household, family expenditures}
		& HC-C, 6.a--6.d & Loans to individuals for household, family and other expenditures \\
		\hline
	\end{tabular}
\end{table}

	\subsection{HC-P 1-4 Family Residential Mortgages in Domestic Offices}
	Schedule HC-P of FR Y-9C deals with residential mortgages which were originated or bought from a third party which are held in the books of the bank. HC-P also looks at a quarterly increase as well and so can serve as a barometer for showing the increase in mortgages and will be a useful aggregate tool to measure mortgage growth in a quarter for BHCs. The mortgages reported in HC-P should be either in the Held for Sale(HFS) or Held for trading(HFT) books of the bank.
	
	While the lines are not very granular the fact that HC-P data represents Quarter over quarter growth makes HC-P a valuable tool for Loan data analytics
	
	\subsection{HC-Q Assets and Liabilities measured at Fair value on a recurring basis}
	
	Schedule HC-Q is for Assets and Liabilities valued at Fair value. The concept of fair value is as per the requirements of ASC Topic 820 \cite{ASC820}. For assets which can be netted as per ASC 210-20 \cite{ASC210}. This schedule has 5 columns. The description of each column is given in the table \eqref{tab:HC-QCol}.
	
	\begin{table}[htbp]
		\centering
		\caption{Schedule HC-Q Reporting Columns (FR Y-9C)}
		\label{tab:HC-QCol}
		\begin{tabular}{|p{1.8cm}|p{6.2cm}|}
			\hline
			\textbf{Column} & \textbf{Description} \\
			\hline
			A & Total fair value of assets and liabilities measured at fair value. \\
			\hline
			B & LESS: Amounts Netted in the Determination of Total Fair Value \\
			\hline
			C & Level 1 Fair Value Measurements \\
			\hline
			D & Level 2 Fair Value Measurements \\
			\hline
			E & Level 3 Fair Value Measurements \\
			\hline
		\end{tabular}
	\end{table}
	
	Table \eqref{tab:FairValueLevelDef} gives the brief definitions of Level 1, Level 2 and Level 3 according to ASC 820 \cite{ASC820}.
	\begin{table}[htbp]
		\centering
		\caption{Fair Value Measurement Hierarchy (FR Y-9C, Schedule HC-Q)}
		\label{tab:FairValueLevelDef}
		\begin{tabular}{|p{1.5cm}|p{2cm}|p{2.5cm}|}
			\hline
			\textbf{Level} & \textbf{Description} & \textbf{Example} \\
			\hline
			Level 1 & Quoted prices in active markets for identical assets or liabilities. & Exchange-traded equity securities \\
			\hline
			Level 2 & Observable inputs other than quoted prices; includes market data for similar instruments. & Agency MBS, corporate bonds \\
			\hline
			Level 3 & Unobservable inputs; valuation relies on internal models or assumptions. & Private equity, complex structured loans \\
			\hline
		\end{tabular}
	\end{table}
	
	The bank will have to report the gross values in columns C, D , E and if there is netting effect the amount saved by netting should be reported in column B. 
	Loans which are Held for Sale are reported in Line 3 and Loans which are held for investment are reported in Line 4. These lines are not very granular but HC-Q has a memo section where more granular loan data is reported with sub-divisions under each loan line reported in the main schedule. The table below gives the comparison between loan data in HC-C and the loan data requested in HC-Q. Given that HC-Q looks at details of assets and liabilities priced at fair value alone, the loan purpose is not concentrated upon instead focusing on collateral and borrower.
	
	\subsubsection{HC-Q Memo}
	HC-Q has also got a memoranda part where the column disaggregation is the same as in HC-Q. With regard to loans, only line Memo line 1.a where the Mortgage Servicing Assets which exceed 25 \% of all other loans reported in HC-Q Line 6 need to be reported.
	
	
\begin{table}[htbp]
	\centering
	\caption{Mapping of HC-Q Memo Loan Categories to HC-C Loan Categories}
	\begin{tabular}{|p{2.5cm}|p{2.5cm}|p{2cm}|}
		\hline
		\textbf{HC-Q Memo (Fair Value)} & \textbf{HC-C (Book Value)} & \textbf{Line Reference} \\
		\hline
		Loans secured by 1–4 family residential properties & Loans secured by 1–4 family residential properties & HC-Q Memo Line 3.a.1,HC-Q Memo Line 4.a.1 $\leftrightarrow$ HC-C Line 1.c.1,HC-C Line 1.c.1 \\
		\hline
		Loans secured by multifamily (5+ units) residential properties & Loans secured by multifamily residential properties & HC-Q Memo Line 3.a.2,HC-Q Memo Line 4.a.2 $\leftrightarrow$ HC-C Line 1.d \\
		\hline
		Loans secured by nonfarm nonresidential properties & Loans secured by nonfarm nonresidential properties & HC-Q Memo Line 3.a.2,HC-Q Memo Line 4.a.2 $\leftrightarrow$ HC-C Line 1.e \\
		\hline
		Construction, land development, and other land loans & Construction, land development, and other land loans & HC-Q Memo Line 4 $\leftrightarrow$ HC-C Line 1.a \\
		\hline
		Loans to depository institutions & Loans to depository institutions & HC-Q Memo Line 4.d $\leftrightarrow$ HC-C Line 2.a ,HC-C Line 2.b\\
		\hline
		Agricultural production loans & Loans to finance agricultural production and other loans to farmers & HC-Q Memo Line 4.d $\leftrightarrow$ HC-C Line 3 \\
		\hline
		Commercial and industrial loans & Commercial and industrial loans & HC-Q Memo Line 4.b $\leftrightarrow$ HC-C Line 4.a,HC-C Line 4.b,HC-C Line 4.c \\
		\hline
		Consumer loans (credit cards, auto, other) & Loans to individuals (consumer loans) & HC-Q Memo Line 4.c $\leftrightarrow$ HC-C Line 6.a,HC-C Line 6.b,HC-C Line 6.c,HC-C Line 6.d \\
		\hline
		Loans to foreign governments and official institutions & Loans to foreign governments and official institutions & HC-Q Memo Line 4.d $\leftrightarrow$ HC-C Line 7 \\
		\hline
		Loans to nondepository financial institutions & Loans to nondepository financial institutions & HC-Q Memo Line 4.d $\leftrightarrow$ HC-C Line 9 \\
		\hline
		All other loans & All other loans (catch-all categories) & HC-Q Memo Line 4.d $\leftrightarrow$ HC-C Line 9.b.1, 9.b.2, 9.b.3 \\
		\hline
	\end{tabular}
\end{table}

\subsection{HC-R Regulatory Capital}
    Schedule HC-R Regulatory Capital deals with the regulatory capital that banks must hold post the Basel III rules. There are two parts to this schedule. The first part is on the Regulatory Capital and Ratios which mostly deal with the numerator part in the Capital to Risk Weighted Assets(RWA) ratios. The second part is on the Risk Weighted Assets which provide comprehensive data on the RWA separated out by the Risk Weights assigned to each trade/transaction.
    
    HC-R Part II provides a comprehensive treatment of loans and leases under the lines . Loans and securitizations are represented in the lines given in table \ref{tab:HCR descriptions}.
    
   \begin{table}[htbp]
   	\centering
   	\caption{Descriptions of Selected Lines in HC-R Part II: Loans / Commitments / Allowance}
   	\label{tab:HCR descriptions}
   	\begin{tabular}{|p{1.5cm}|p{5.5cm}|}
   		\hline
   		\textbf{Line} & \textbf{Description} \\
   		\hline
   		4a & Loans and leases held for sale: Residential mortgage exposures. \\
   		\hline
   		4b & Loans and leases held for sale: High volatility commercial real estate exposures. \\
   		\hline
   		4c & Loans and leases held for sale: Exposures past due 90 days or more or on nonaccrual. \\
   		\hline
   		5a & Loans and leases held for investment: Residential mortgage exposures. \\
   		\hline
   		5b & Loans and leases held for investment: High volatility commercial real estate exposures. \\
   		\hline
   		5c & Loans and leases held for investment: Exposures past due 90 days or more or on nonaccrual. \\
   		\hline
   		6 & LESS: Allowance for loan and lease losses (to be deducted). \\
   		\hline
   		18(a,b) & Unused commitments (excluding unconditionally cancelable ones):  
   		\newline 18(a) Unused commitments with original maturity of one year or less  
   		\newline 18(b) Unused commitments with original maturity of more than one year \\
   		\hline
   		19 & Unused commitments that are unconditionally cancelable \\
   		\hline
   	\end{tabular}
   \end{table}
   
    
    Each line has MDRMs for the applicable Risk Weights in addition to the amount reported in Schedule HC for the same MDRM in column A and adjustments to the amount reported in column A in column B. Adjustments can take place when there are components to the line item which should not be risk weighted and so would not fall in any of the risk buckets like gain on securitization exposures. The RWA buckets in HC-R Part II are 0\%, 2\%, 4\%,10\%,20\%,50\%,100\%,150\%,250\%, 300\%,600\% and 1250\%. Not all exposures will have all the RWA percentages and so some MDRMs might be skipped for certain RWA percentages.
    
      In case of on balance sheet exposures like loans, the risk weight depends upon the on whether the exposures is guaranteed by the US Government or a US Government Sponsored Entity(GSE) like Fannie Mae. In case there is no guarantee. In case there is no guarantee, it depends upon the counterparty whether the counterparty is a US Public Sector Enterprise(PSE), a US Bank or a Foreign Bank or a Retail Entity. There might be product specific classifications for Risk Weight allocation like for Retail exposures, the Risk Weight may depend on if the lending is secured(like a home mortgage) or unsecured. \cite{PolkB3}
      
      As a general rule, the total amount reported in schedule HC for a line item in HC-R part II (reported in column A of HC-R Part II) should be equal to the sum of the amounts reported in the other columns from column B till Column Q or column R as applicable for lines from 1 to 10. In case of securitizations in Line 9 and 10 of HC-R Part II since the only Risk Weight applicable is 1250\%, the other columns are not required and so removed off the reporting form. However based on the approach taken for securitization like Simplified Supervisory Formula Approach(SSFA) or Gross Up approach, columns T and U are reported for lines 9 and 10. A discussion about SSFA and Gross Up approach is beyond the scope of this paper.
      
      Derivatives and off balance sheet items in HC-R Part II from lines 12 till 21, column B is filled with the Credit Equivalent amount which is obtained by multiplying the amount reported in column A with the appropriate Credit Conversion Factor(CCF) provided in the form. 
      
      Table \ref{tab:HCR RW} gives a general summary of reporting for each column for each of the loan related lines in Schedule HC-R Part II.
	

\begin{table}[htbp]
	\centering
	\caption{Applicable Risk Weights for Selected Lines in Schedule HC-R Part II}
	\label{tab:HCR RW}
	\begin{tabular}{|p{3cm}|p{1.5cm}|p{1.5cm}|}
		\hline
		\textbf{Line Item} & \textbf{Adjustment Col B Nature} & \textbf{Applicable Risk Weights} \\
		\hline
		4a. Loans and leases held for sale: Residential mortgages & \multirow{7}{1.5cm}{\centering Exposures for which no risk weight can be assigned.} & 0\%,20\%,50\% ,100\% \\
		\cline{1-1} \cline{3-3}
		4b. Loans and leases held for sale: High volatility commercial real estate (HVCRE) &  & 0\%,20\%,50\% ,100\%,150\% \\
		\cline{1-1} \cline{3-3}
		4c. Loans and leases held for sale: Past due $\ge$90 days or nonaccrual &  & 0\%,2\%,4\% ,20\% ,50\% ,100\%,150\% \\
		\cline{1-1} \cline{3-3}
		5a. Loans and leases held for investment: Residential mortgages &  & 0\%,20\%,50\% ,100\% \\
		\cline{1-1} \cline{3-3}
		5b. Loans and leases held for investment: High volatility commercial real estate (HVCRE) &  & 0\%,20\% ,50\%,100\% ,150\% \\
		\cline{1-1} \cline{3-3}
		5c. Loans and leases held for investment: Past due $\ge$90 days or nonaccrual &  & 0\%,2\%,4\% ,20\% ,50\%,100\% ,150\% \\
		\cline{1-1} \cline{3-3}
		6. Less: Allowance for loan and lease losses (deduction) &  & Deduction (not risk-weighted) \\
		\hline
		9a. On balance sheet securitization exposures : Held to Maturity securities & \multirow{4}{1.5cm}{\centering Depends on the Risk weighting approach and the AOCI Opt out election} & 1250\% \\
		\cline{1-1} \cline{3-3}
		9b. On balance sheet securitization exposures : Available for sale securities &  & 1250\% \\
		\cline{1-1} \cline{3-3}
		9c. On balance sheet securitization exposures : All other on balance sheet and securitization exposures &  & 1250\% \\
		\cline{1-1} \cline{3-3}
		10. Off balance sheet securitization exposures &  & 1250\% \\
		\cline{1-1} \cline{3-3}
		\hline
		18a. Unused commitments with maturity $<$1 year (non-cancelable) &\multirow{3}{1.5cm}{\centering Credit Equivalent Amount of commitment}  & 0\%,2\%,4\% ,20\% ,100\%,150\% \\
		\cline{1-1} \cline{3-3}
		18b. Unused commitments with maturity $>$1 year (including certain ABCP conduits) &  & 0\%,2\%,4\% ,20\% ,100\%,150\% \\
		\cline{1-1} \cline{3-3}
		19. Unused commitments unconditionally cancelable &  & No risk weights \\
		\hline
	\end{tabular}
	\end{table}

    The concept of a credit equivalent amount arises because off balance sheet exposures are reported as notional balances. Conversion of notional balances to the equivalent amounts to be reported in a form is an arbitrary exercise. The Fed does this in the FR Y-9C by requiring banks to multiply the notional amounts by a Credit Conversion Factor(CCF) which gives the Credit Equivalent amount. The CCF is usually a number between 0 and 1.
    
    \subsubsection{HC-R Memo}
    	The memoranda section for HC-R covers only the allowance for Purchased Credit Deteriorated(PCD) loans which aligns broadly with theme of HI-C Disaggregated data on allowance for loans and lease losses. This is to be reported in Memo Line 5 which is broken down into the PCD losses for HFI, HTM and other types loans which are measured at amortized cost.
    	
    \subsection{HC-S Servicing , Securitization and Asset Sale Activities}
       Schedule HC-S deals with Securitizations. In this schedule there are seven columns each of which pertain to the product securitized. The seven products which are used for categorization in securitization are
       
       \begin{enumerate}
       		\item 1-4 Family Residential Loans
       		\item Home Equity lines
       		\item Credit Card Receivables
       		\item Auto Loans
       		\item Other consumer loans
       		\item Commercial and Industrial Loans
       		\item  All other loans and leases and other assets
       \end{enumerate}
       
       Since securitization is primarily an off balance sheet line item, most of the data reported here will be mutually exclusive to other schedules of FR Y-9C. The exceptions will be data reported that corresponds Schedule HC-R and HC-L which also deal with off balance sheet items. However the overall theme of each line can be mapped to a schedule or Line in other schedules of FR Y-9C.
       
       Table \eqref{tab:hcs_hcc_mapping} gives the mapping between each column in HC-S to a line HC-C. The table looks at the theme of the corresponding line in other schedule alone and does not look at whether the line item is an on balance sheet item or an off balance sheet item. Given that HC-C rows are all based on upon the Loan product based on Collateral, Borrower and purpose, this would be the right categorization.
       
       
       \begin{table}[htbp]
       	\centering
       	\caption{Mapping of Schedule HC-S Columns to Schedule HC-C Line Items}
       	\label{tab:hcs_hcc_mapping}
       	\begin{tabular}{|p{3cm}|p{1.6cm}|p{2.5cm}|}
       		\hline
       		\textbf{HC-S Column and Asset Category} & \textbf{Corresponding HC-C Line Item} & \textbf{HC-C Description} \\
       		\hline
       		Column A :1-4 Family Residential Loans & HC-C, Item 1.c & Loans secured by 1-4 family residential properties \\
       		\hline
       		Column B : Home Equity Lines & HC-C, Item 1.c.(1) & Revolving, open-end loans secured by 1-4 family residential properties and extended under lines of credit \\
       		\hline
       		Column C : Credit Card Receivables & HC-C, Item 6.a & Credit cards \\
       		\hline
       		Column D : Auto Loans & HC-C, Item 6.c & Automobile loans \\
       		\hline
       		Column E :  Other Consumer Loans & HC-C, Item 6.d & Other consumer loans (includes single payment, installment, and all student loans) \\
       		\hline
       		Column F : Commercial and Industrial Loans & HC-C, Item 4 & Commercial and industrial loans \\
       		\hline
       		Column G : All Other Loans, All Leases, and All Other Assets & Multiple HC-C Items & Various loan categories including: Items 1.a-1.b (real estate), 2 (depository institutions), 3 (agricultural), 7 (foreign governments), 9 (nondepository financial institutions), and 10 (lease financing receivables) \\
       		\hline
       	\end{tabular}
       \end{table}
       
       In terms of the lines of reporting, HC-S is divided into three sections , Securitization Activites, Securitization Facilities sponsored and Asset Sale activities. The relationship between HC-S and other schedules reporting loans and commitments like HC-L,HC-N, HC-P,HC-Q and HC-R can described as shown in \eqref{tab:hcs_relationships}. Since securitization is an off balance sheet item, most of the relationships are related only in the nature of the product .
       
       \begin{table}[htbp]
       	\centering
       	\caption{Relationship Between Schedule HC-S and Other FR Y-9C Schedules}
       	\label{tab:hcs_relationships}
       	\begin{tabular}{|p{1.5cm}|p{2cm}|p{1.5cm}|p{2cm}|}
       		\hline
       		\textbf{HC-S Item} & \textbf{HC-S Description} & \textbf{Related Schedule \& Item} & \textbf{Relationship Description} \\
       		\hline
       		\multicolumn{4}{|c|}{\textbf{HC-S to HC-L (Off-Balance Sheet Items)}} \\
       		\hline
       		Item 3 , Item 10 and Memo Item 3.b & Unused commitments to provide liquidity to consolidated which have been sold and securtized & HC-L, Item 1 & Liquidity commitments to securitization structures are reported as unused commitments \\
       		\hline
       		Item 9 and Memo Item 3.a & Credit enhancements to other institutions' structures & HC-L, Item 2 and Item 3 & Credit enhancements via standby letters of credit both Financial and Performance \\
       		\hline
       		\multicolumn{4}{|c|}{\textbf{HC-S to HC-N (Past Due and Nonaccrual)}} \\
       		\hline
       		Item 4.a and 4.b & Assets 30-89 days past due and assets 90+ days past due & HC-N, Items 1-8, Column A and B & Past due amounts for securitized loans still on balance sheet \\
       		\hline
       		\multicolumn{4}{|c|}{\textbf{HC-S to HI-C (Charge off and Recoveries in loans and leases)}} \\
       		\hline
       		Item 5.a & Charge-offs on securitized assets & HI-C Lines 1 to 9 column A & Year-to-date charge-offs on retained interests \\
       		\hline
       		Item 5.b & Recoveries on securitized assets & HI-C Lines 1 to 9 column B & Year-to-date recoveries on retained interests \\
       		\hline
       		\multicolumn{4}{|c|}{\textbf{HC-S to HC-P (Mortgage Banking Activities)}} \\
       		\hline
       		Item 1 and Item 6, Column A & Outstanding 1-4 family residential mortgages securitized & HC-P, Item 3 and 4 & Mortgages sold during quarter  \\
       		\hline
       		Memo Item 2.a and 2.b & 1-4 family mortgages serviced with and without recourse & HC-P, Items 1-2 and HC-P Line 5& Originated/ purchased mortgages with servicing retained. Servicing income from mortgages. \\
       		\hline
       		\multicolumn{4}{|c|}{\textbf{HC-S to HC-Q (Fair Value Measurements)}} \\
       		\hline
       		Item 6 & Ownership interests carried as loans & HC-Q, Item 4 & Loans held for investment at fair value \\
       		\hline
       		Items 1-12 & Credit enhancements and retained interests & HC-Q, Memo Item 2.a & Loan commitments at fair value \\
       		\hline
       		\multicolumn{4}{|c|}{\textbf{HC-S to HC-R (Regulatory Capital)}} \\
       		\hline
       		Item 2 & Maximum credit exposure from recourse/ enhancements & HC-R, Part II, Item 15 & Retained recourse on sold obligations (100\% risk weight) \\
       		\hline
       		Items 9, 12 & Credit enhancements provided & HC-R, Part II, Item 12 & Financial standby letters of credit (risk-weighted) \\
       		\hline
       		Item 1,6 and Memo Item 3 & Outstanding securitized assets & HC-R, Part II, Items 9-10 & On- and off-balance sheet securitization exposures \\
       		\hline
       	\end{tabular}
       \end{table}
       
       \subsubsection{HC-S Memo}
       
      	HC-S Memo also has some details about loans. Line 1 requires reporting of servicing done for loans for other entities for 1-4 loans with or without recourse and loans under foreclosure. There is also a catch all bucket where other financial assets serviced by the BHC need to be reported. In Memo Line 4 the Credit card servicing charges and fees which have been reported as part of original HC-S Line 1 Col C need to be reported.
      	
	\subsection{HC-V Variable interest entities}
	
		A Variable Interest Entity(VIE) is an entity separate from the bank in which the bank has got a significant interest. One of the reasons a VIE is created is to support the securitization activities of the bank. Schedule HC-V of FR Y-9C is report the activities of the VIEs in which the bank has got a significant interest.
		
		The definition of a VIE is as given in ASC 810 \cite{ASC810}. ASC 810 of FASB gives the defintion of a Variable Interest entity and when the VIE should be reported in the consolidated financial statements as well. The definition of a VIE can be defined as an entity in which the bank or any other parent entity has an interest but lacks the following.
		
		\begin{enumerate}
			\item The power through voting rights to direct the activity of the entity
			\item The obligation to absorb the losses of the entity
			\item The right to receive the residual returns of the entity during liquidation
		\end{enumerate}
		
		The concept of a VIE is unique to the US GAAP accounting rules and is not present in the International Financial Reporting Standards (IFRS) . The FASB requires consolidation of VIEs in order to ensure that off balance sheet entities are also reported in the consolidated financial statements of the financial entity.
		
		VIEs are usually used in the banking industry for the formation of the Special Purpose Vehicles(SPV) independent of the parent bank for the purposes of securitization.
		
		The difference in data reported in HC-S and HC-V is that in HC-S , loans securitized and sold and therefore which are not on the balance sheet of the BHC need to be reported while in HC-V the financial information of the VIE need to be reported. In case the BHC has sold the loans to a SPV which qualifies as VIE, the details of the loans sold will have to be reported in HC-S and the details of the VIE to which the loans have been sold to has to be reported in HC-V.
		
		Since a VIE can exist for entities outside of securitization as well, the concept of a VIE need not completely pertain to a SPV alone. So with regard to loan details of a VIE, line 1.c pertains to the portion of assets of the VIE which consist of Loans and Leases which are either held for investment(HFI) or held for sale (HFS).

	The overall picture for loans data present in various schedules of FR Y-9C is provided in the picture \eqref{LoanHier}.
	
		
	\begin{figure*}[t!]
		\centering
		\includegraphics[width=\textwidth,height=0.9\textheight,keepaspectratio]{combined_flowchart_TB.png}
		\caption{Loan Data Representation in FR Y-9C}
		\label{LoanHier}
	\end{figure*}
	
	
\section{Comparison of data in FR Y-9C with other reports submitted to the Federal Reserve}
	The Federal Reserve collects information in other forms at a balance sheet level other than the FR Y-9C. Table \eqref{tab:FedForms} provides the details of some of the important forms.
	
	\begin{table}[htbp]
		\caption{Federal Reserve Reporting Forms}
		\label{tab:FedForms}
		\centering
		\begin{tabular}{|l|p{7cm}|}
			\hline
			\textbf{Form Name} & \textbf{Description} \\
			\hline
			FR Y-9LP       & Parent company only financial statements for bank holding companies. \\
			\hline
			FR Y-15        & Systemic risk report \\
			\hline
			FR Y-11        & Financial Statements of U.S. Nonbank Subsidiaries of U.S. Holding Companies \\
			\hline
			FFIEC-002      & Report of assets and liabilities of U.S. branches and agencies of foreign banks. \\
			\hline
			FFIEC-031      &  Consolidated report of condition and income for banks with domestic and foreign offices. \\
			\hline
			FFIEC-041      & Consolidated report of condition and income for banks with domestic offices only.  \\
			\hline
		\end{tabular}
		\label{tab:fed_forms}
	\end{table}
	
	All the forms described in Table \eqref{tab:FedForms} have got data aggregated and standarized into standard MDRMs. This section will talk briefly on the merits and demerits of usage of these forms instead of FR Y-9C.
	
	\subsection{FR Y-9LP Parent only financial statements for BHCs}
	
		The FR 9-LP Parent only financial statements for BHCs looks at the financials of the Parent Holding company alone whereas the FR Y-9C looks at the parent holding company at a consolidated level. Because the Parent financial company is looked at a standalone level, the amount of detail required by the Fed is considerably smaller for FR Y-9LP when compared to FR Y-9C. Table \eqref{tab:FRY9LP Schedules} gives the schedules of the FR Y-9LP. 
		
		\begin{table}[htbp]
			\caption{FR Y-9LP Schedules}
			\label{tab:FRY9LP Schedules}
			\centering
			\begin{tabular}{|l|p{6cm}|}
				\hline
				\textbf{Schedule Name} & \textbf{Short Description} \\
				\hline
				Schedule PI & Parent Company Only Income Statement \\
				\hline
				Schedule PI-A & Cash Flow Statement \\
				\hline
				Schedule PC & Parent Company Only Balance Sheet \\
				\hline
				Schedule PC-A & Investments in Subsidiaries and Associated Companies \\
				\hline
				Schedule PC-B & Memoranda (additional balance sheet and related disclosures) \\
				\hline
				Notes & Notes to the Parent Company Only Financial Statements \\
				\hline
			\end{tabular}
		\end{table}
		
		As can be seen from FR Y-9LP schedules, the schedules are considerably less in FR Y-9LP when compared to FR Y-9C since only the assets ,liabilities and cash flow of the Parent alone need to be reported. While loans data can be gleaned from FR Y-9LP in PC Schedule line 4, the details required to be provided are not so detailed as Schedule HC-C in FR Y-9C. However should the need arise to get details of loans provided directly from the Parent BHC, the FR Y-9LP can provide required details. 
		
		Line 4.a.1 and 4.a.2 concentrate on the whether the counterparty is US Based or not. Line 4.b delves into the unearned income. Line 4.d gets into the lease financing part. Line 4.e talks about the allowances set aside at a Parent BHC level. 
		
		For further detail the FR Y-9LP form and instructions are available at \cite{FRY9LP}.
		
		Given that FR Y-9C and FR Y-LP were the same report till 1986, the FR Y-9LP data is also available in Public similar to FR Y-9C.
		
		\subsection{FR Y-15 Systemic Risk Report}
		    The FR Y-15 Systemic risk report is to be filed by BHCs, IHCs and SLHCs which have assets over \$ 100 billion. This report is part of the enhanced macro-prudential supervision required as part of the Dodd Frank Act by the Fed for applicable banks. The Fed measures six aspects of a bank's footprint 
		    
		    \begin{enumerate}
		    	\item   Size
		    	\item Interconnectedness
		    	\item Substitutability
		    	\item Complexity
		    	\item Cross-jurisdictional activity
		    	\item Short-term wholesale funding
		    \end{enumerate}
		    
		    All of these are provided in FR Y-15 \cite{FRY15}. FR Y-15 actually is a combination of two separate report. All the above numbered factors are examined for BHCs headquartered in US from Schedules A to G and they are examined separately for Intermediate Holding companies(IHCs) of Foreign Banking Organizations(FBOs) from Schedules H to N. 
		    
		    From a loan analysis perspective, the FR Y-15 provides very limited opportunity with a handful of MDRMs providing data which may be useful for loan analysis.The few MDRMs in Schedule B - Interconnectedness and Schedule I - FBO Interconnecteness deal with Inter bank lending alone and look at the various forms of interbank lending like commitments and Secured Financial Transactions(SFTs) like repos. However there are MDRMs which report bank counterparty lending activities. Table \ref{tab:fry15_schedules} provides the details of each schedule in FR Y-15.
		    
		    \begin{table}[htbp]
		    	\centering
		    	\caption{FR Y-15 Systemic Risk Report Schedules}
		    	\label{tab:fry15_schedules}
		    	\begin{tabular}{|p{2.5cm}|p{6cm}|}
		    		\hline
		    		\textbf{Schedule Name} & \textbf{Schedule Description} \\
		    		\hline
		    		\multicolumn{2}{|c|}{\textbf{Standard Schedules (All Filers)}} \\
		    		\hline
		    		Schedule A & Size Indicator \\
		    		\hline
		    		Schedule B & Interconnectedness Indicators \\
		    		\hline
		    		Schedule C & Substitutability Indicators \\
		    		\hline
		    		Schedule D & Complexity Indicators \\
		    		\hline
		    		Schedule E & Cross-Jurisdictional Activity Indicators \\
		    		\hline
		    		Schedule F & Ancillary Indicators \\
		    		\hline
		    		Schedule G & Short-Term Wholesale Funding Indicator \\
		    		\hline
		    		\multicolumn{2}{|c|}{\textbf{FBO-Specific Schedules}} \\
		    		\hline
		    		Schedule H & FBO Size Indicator \\
		    		\hline
		    		Schedule I & FBO Interconnectedness Indicators \\
		    		\hline
		    		Schedule J & FBO Substitutability Indicators \\
		    		\hline
		    		Schedule K & FBO Complexity Indicators \\
		    		\hline
		    		Schedule L & FBO Cross-Jurisdictional Activity Indicators \\
		    		\hline
		    		Schedule M & FBO Ancillary Indicators \\
		    		\hline
		    		Schedule N & FBO Short-Term Wholesale Funding Indicator \\
		    		\hline
		    		\multicolumn{2}{|c|}{\textbf{Optional Schedule}} \\
		    		\hline
		    		Optional Narrative & Optional Narrative Statement (750 character limit for public disclosure regarding reported values) \\
		    		\hline
		    	\end{tabular}
		    \end{table}
		  
		  \subsection{FR Y-11 Financial Statements of U.S. Nonbank Subsidiaries of U.S. Holding Companies}
		  
		  FR Y-11 report deals with the financial nonbank subsidiaries of a BHC\cite{FRY11}. The BHC files the FR Y-9C at a consolidated level , while the FR Y-11 is at a subsidiary level. The BHC can have a number of nonbank and banking subsidiaries. The banking subsidiaries will be required to file the FFIEC-031 or FFIEC-041 and the non bank subsidiaries will have to file the FR Y-11. The details of the reporting structure between FR Y-9C, FR Y-11 , FFIEC-031 and FFIEC-041 are provided in fig.
		  
		  The definition of a non-bank subsidiary according the Fed has been provided in the FR Y-11 instructions \cite{FRY11} with the exemptions mentioned in a separate section in the instructions as well. Deposit taking entities of a BHC should not file the FR Y-11 and instead file the FFIEC-031 or FFIEC-041 as applicable. Other instructions include FBO filers which file reports like FR-2314 or FR-Y20.
		  
		    
		  Since the FR Y-11 is filed at a subsidiary level, the FR Y-11 numbers for the majority of cases will be available in FR Y-9C at a consolidated as well. But given that the FR Y-11 is filed at a subsidiary level, the thresholds for filing are well below the \$ 3 billion for FR Y-9C at \$ 500 million of assets at a parent level with the subsidiary having assets of greater than \$ 1 billion or off balance sheet activities of greater than \$ 5 billion or equity capital of subsidiary more than 5 \% of parent or revenue of subsidiary more than \% 5 percent of parent. 
		  
		  FR Y-11 needs to be filed annually for those subsidiaries whose assets are between \$ 250 million and \$500 million and on a quarter basis for subsidiaries whose assets are above \$500 million. The annual filers file an abbreviated report where a lot of MDRMs are left blank.
		  
		  Because of the lower threshold the level of detail required at FR Y-11 is also lesser than FR Y-9C in almost all scenarios. Table \eqref{tab:fry11_schedules} provides the schedules of FR Y-11
		  
		  \begin{table}[htbp]
		  	\centering
		  	\caption{FR Y-11 Financial Statements of U.S. Nonbank Subsidiaries Schedules}
		  	\label{tab:fry11_schedules}
		  	\begin{tabular}{|p{1.5cm}|p{6cm}|}
		  		\hline
		  		\textbf{Schedule} & \textbf{Schedule Description} \\
		  		\hline
		  		Schedule IS & Income Statement (calendar year-to-date) \\
		  		\hline
		  		Schedule IS-A & Changes in Equity Capital \\
		  		\hline
		  		Schedule IS-B & Changes in Allowance for Credit Losses \\
		  		\hline
		  		Schedule BS & Balance Sheet \\
		  		\hline
		  		Schedule BS-A & Loans and Lease Financing Receivables (exclude balances with related institutions) \\
		  		\hline
		  		Schedule BS-M & Memoranda (supplemental balance sheet information including loans to non-U.S. addressees, servicing portfolio, intangible assets, and trading account details) \\
		  		\hline
		  		Notes & Notes to the Financial Statements \\
		  		\hline
		  	\end{tabular}
		  \end{table}
		  
		  With regard to Loan data, the allowance change details are provided in Schedule IS-B. However unlike FR Y-9C which has a HI-C Schedule for providing the disaggregated data on allowances on loan and lease losses, there is no disaggregation schedule in FR Y-11. In a similar manner, Schedule BS-A which can be considered the only detail level product schedule at the balance sheet level in FR Y-11 is entirely dedicated to loans. However the level of detail required for providing detail is considerably lower when compared the Schedule HC-C of FR Y-9C.
		  
		  Also while FR Y-11 data can be requested for individual banks with a Freedom of Information (FOIA) request to the Federal Reserve, it is not as freely available in the FFIEC website like FR Y-9C which makes it a dampener to do analysis based on FR Y-11 data.
		  
		  \subsection{Call reports FFIEC-031, FFIEC-041 and FFIEC-051}
		  		  
		  The reports FFIEC-031, FFIEC-041 and FFIEC-051 are called call reports which provide the details of the individual banks within a bank holding company. These reports are closely aligned as can be seen from their descriptions
		  
		  \begin{itemize}
		  	\item FFIEC-031 - Consolidated Reports of Condition and Income for 
		  	a Bank with Domestic and Foreign Offices
		  	\item FFIEC-041 - Consolidated Reports of Condition and Income for 
		  	a Bank with Domestic Offices Only
		  	\item FFIEC-051 - Consolidated Reports of Condition and Income for a Bank with Domestic Offices Only and Total Assets Less than \$5 Billion
		  \end{itemize}
		  
		  All of these reports have to be submitted quarterly and have schedules which are similar in nature to FR Y-9C. The nature of the reports are similar to each other and as a result FFIEC-031 and FFIEC-041 have the same instruction set with a few MDRMs lesser in FFIEC-041 since data pertaining to foreign offices need not be reported.
		  
		  \subsubsection{FFIEC-031 and FFIEC-041}
		      The FFIEC-031 \cite{FFIEC031} and FFIEC-041 \cite{FFIEC041} reports will have to be filed by banks which are classified as a national bank, state member bank, insured state nonmember bank or a savings association. Because this applies only for the bank classifications mentioned, a holding company which does not directly involve in banking activity need not file the call reports of FFIEC-031 or FFIEC-041.So this can be thought of as report filed by the banking(deposit taking) subsidiaries of the BHCs alone. 
		      
		      Call reports have some additional schedules which are not present in FR Y-9C like RI-D and RI-E. Table \ref{tab:FFIEC031041_Schedules} provides the additional schedules present in FFIEC-031 and FFIEC-041 which are not present in FR Y-9C. 
		      
		      \begin{table}[htbp]
		      	\caption{Schedules Unique to FFIEC-031 and FFIEC-041}
		      	\label{tab:FFIEC031041_Schedules}
		      	\centering
		      	\begin{tabular}{|p{0.75cm}|p{4.5cm}|p{0.75cm}|}
		      		\hline
		      		\textbf{Schedule Name} & \textbf{Schedule Header} & \textbf{Present In} \\
		      		\hline
		      		RC-O & Other Data for Deposit Insurance Assessments & Both \\
		      		\hline
		      		RC-T & Fiduciary and Related Services & Both \\
		      		\hline
		      		RI-C & Disaggregated Data on Allowances for Credit Losses & Both \\
		      		\hline
		      		RI-D & Income from Foreign Offices & FFIEC-031 only \\
		      		\hline
		      	\end{tabular}
		      \end{table}
		      
		      The FFIEC-031 and FFIEC-041 mainly differ in only one small aspect that FFIEC-031 is for banks with both domestic and foreign offices while FFIEC-041 is for banks with domestic offices only. Because of this FFIEC-041 does not have the schedule RI-D.With regard to thresholds for filing the FFIEC-031 has a threshold of \$ 100 billion. If a bank has assets less than \$ 100 billion but has foreign offices, the bank has to still file the FFIEC-031. The threshold for FFIEC-041 is between \$ 5 billion and \$ 100 billion. Unlike the FR Y-11 which will have to be filed at an individual subsidiary level, the Call reports will have to be filed at a consolidated level meaning that if there are subsidiaries under the bank, they will also have to be reported in the Call report of the bank. 
		      
		      A significant omission when FR Y-9C is considered is that the call reports do not contain the equivalent of schedule HC-H which is to measure the interest sensitivity of balances of the bank. 
		  
		  	\subsubsection{FFIEC-051 - Consolidated Reports of Condition and Income for a Bank with Domestic Offices Only and Total Assets Less than \$5 Billion}
		  	
		  	The FFIEC-051 is for banks with assets less than \$ 5 Billion and without any foreign offices. FFIEC-051 is reported quarterly like the other call report but given the low asset threshold the FFIEC-051 reporting burden is significantly reduced and a lot of schedules which need to be reported in the other call reports FFIEC-031 and FFIEC-041 need not be reported in FFIEC-051.
		  	
		  	Table \eqref{tab:FFIEC_SchedulesNot051} provides the set of schedules which are not present in FFIEC-051 but which are present in FFIEC-031 and FFIEC-041.
		  	
		  	\begin{table}[htbp]
		  		\caption{Schedules in FFIEC-031/041 But Not in FFIEC-051}
		  		\label{tab:FFIEC_SchedulesNot051}
		  		\centering
		  		\begin{tabular}{|l|p{4cm}|l|}
		  			\hline
		  			\textbf{Schedule} & \textbf{Schedule Description} & \textbf{Present In} \\
		  			\hline
		  			RC-A & Cash and Balances Due From Depository Institutions (detailed variant) & FFIEC-031 only \\
		  			\hline
		  			RC-D& Trading Assets and Liabilities&Both\\
		  			\hline
		  			RC-H & Selected Balance Sheet Items for Domestic Offices & FFIEC-031 only \\
		  			\hline
		  			RC-I & Assets and Liabilities of IBFs & FFIEC-031 only \\
		  			\hline
		  			RC-M & Memoranda (with additional items, e.g., transactional internet banking, captive insurers) & Both \\
		  			\hline
		  			RC-N & Past Due and Nonaccrual Loans, Leases, and Other Assets (full version) & Both \\
		  			\hline
		  			RC-P & 1-4 Family Residential Mortgage Banking Activities & Both \\
		  			\hline
		  			RC-Q & Assets/Liabilities at Fair Value (recurring basis, full version) & Both \\
		  			RC-S & Servicing, Securitization, and Asset Sale Activities & Both \\
		  			\hline
		  			RC-V & Variable Interest Entities (VIEs) details & Both \\
		  			\hline
		  			RI-D & Income from Foreign Offices & FFIEC-031 only \\
		  			\hline
		  		\end{tabular}
		  	\end{table}
		  	
		  	
\section{Plan of the paper}
1. Introduction

Motivation: Why analyze loans at the BHC level?

Role of BHCs in U.S. credit markets.
Importance of supervisory datasets for researchers \& policymakers.

Introduce FR Y-9C (public, quarterly, rich coverage) as the focal dataset.

Contribution of this review: summarizing what loan analytics are possible from FR Y-9C and highlighting research directions.

2. Overview of the FR Y-9C Dataset

Origin and regulatory purpose of FR Y-9C.

Coverage: which institutions file it, reporting frequency.

Loan-related schedules:

HC-C (Loans and Lease Financing Receivables)

HC-N (Past Due and Nonaccrual Loans)

HC-L (Derivatives and Off-Balance Sheet Items)

Related income statement data (HI-B interest income, provisions).

Comparison with other data sources (Call Reports, FR Y-14Q, FFIEC 002).

Strengths and limitations (public availability vs. lack of loan-level detail).

3. Analytical Themes in Loan Portfolio Research

(Each subsection reviews existing methods, stylized facts, and what FR Y-9C enables)

3.1 Loan Composition and Growth
Trends across C\&I, CRE, consumer, agricultural, etc.

Concentration vs. diversification.

Business cycle sensitivity of loan growth.

3.2 Credit Risk and Asset Quality

Nonperforming loans, charge-offs, and provisioning.

Loan loss reserves as a measure of expected credit loss.

Stress-period dynamics (e.g., 2008 crisis, COVID-19).

3.3 Profitability and Loan Pricing

Net interest income and yield analysis.

Loan spreads inferred indirectly from interest income vs. loan balances.

Cross-sectional variation by BHC size/class.

3.4 Capital, Liquidity, and Loan Supply

Interaction between capital adequacy and loan growth.

Liquidity positions and loan expansion/contraction.

Links to macroprudential policies.

3.5 Systemic Risk and Interconnectedness

Concentration of lending across sectors.

Role of large vs. small BHCs in systemic credit provision.

Early warning signals from FR Y-9C aggregates.

4. Methodologies for Loan Analytics

Descriptive/statistical analysis (ratios, growth rates, trend decomposition).

Econometric approaches: panel regressions, dynamic models.

Stress-testing style approaches using FR Y-9C data proxies.

Machine learning applications for risk classification.

Comparisons to loan-level data: What’s possible with aggregate data vs. FR Y-14Q.

5. Policy and Supervisory Applications

How regulators use FR Y-9C to monitor loan quality.

Applications to CCAR/DFAST stress testing.

Use in financial stability monitoring (aggregate lending conditions).

Implications for macroprudential vs. microprudential supervision.

6. Limitations of FR Y-9C for Loan Analytics

Lack of borrower-level detail.

Challenges in sectoral disaggregation (some categories broad).

Time lags and quarterly frequency.

Comparisons with richer supervisory datasets (Y-14Q, confidential Fed datasets).

7. Future Research Directions

Linking FR Y-9C with other datasets (e.g., Call Reports, Y-15 systemic risk data, market data).

Improving loan risk modeling with aggregate vs. micro data.

Cross-country comparisons of supervisory reporting.

Potential role of RegTech and data standardization for future analytics.

8. Conclusion

Summarize key takeaways from the review.

Emphasize FR Y-9C’s role as a public and accessible supervisory dataset.

Call for continued innovation in loan analytics using supervisory filings.
\subsection{Abbreviations and Acronyms}\label{AA}
Define abbreviations and acronyms the first time they are used in the text, 
even after they have been defined in the abstract. Abbreviations such as 
IEEE, SI, MKS, CGS, ac, dc, and rms do not have to be defined. Do not use 
abbreviations in the title or heads unless they are unavoidable.

\subsection{Units}
\begin{itemize}
\item Use either SI (MKS) or CGS as primary units. (SI units are encouraged.) English units may be used as secondary units (in parentheses). An exception would be the use of English units as identifiers in trade, such as ``3.5-inch disk drive''.
\item Avoid combining SI and CGS units, such as current in amperes and magnetic field in oersteds. This often leads to confusion because equations do not balance dimensionally. If you must use mixed units, clearly state the units for each quantity that you use in an equation.
\item Do not mix complete spellings and abbreviations of units: ``Wb/m\textsuperscript{2}'' or ``webers per square meter'', not ``webers/m\textsuperscript{2}''. Spell out units when they appear in text: ``. . . a few henries'', not ``. . . a few H''.
\item Use a zero before decimal points: ``0.25'', not ``.25''. Use ``cm\textsuperscript{3}'', not ``cc''.)
\end{itemize}

\subsection{Equations}
Number equations consecutively. To make your 
equations more compact, you may use the solidus (~/~), the exp function, or 
appropriate exponents. Italicize Roman symbols for quantities and variables, 
but not Greek symbols. Use a long dash rather than a hyphen for a minus 
sign. Punctuate equations with commas or periods when they are part of a 
sentence, as in:
\begin{equation}
a+b=\gamma\label{eq}
\end{equation}

Be sure that the 
symbols in your equation have been defined before or immediately following 
the equation. Use ``\eqref{eq}'', not ``Eq.~\eqref{eq}'' or ``equation \eqref{eq}'', except at 
the beginning of a sentence: ``Equation \eqref{eq} is . . .''

\subsection{\LaTeX-Specific Advice}

Please use ``soft'' (e.g., \verb|\eqref{Eq}|) cross references instead
of ``hard'' references (e.g., \verb|(1)|). That will make it possible
to combine sections, add equations, or change the order of figures or
citations without having to go through the file line by line.

Please don't use the \verb|{eqnarray}| equation environment. Use
\verb|{align}| or \verb|{IEEEeqnarray}| instead. The \verb|{eqnarray}|
environment leaves unsightly spaces around relation symbols.

Please note that the \verb|{subequations}| environment in {\LaTeX}
will increment the main equation counter even when there are no
equation numbers displayed. If you forget that, you might write an
article in which the equation numbers skip from (17) to (20), causing
the copy editors to wonder if you've discovered a new method of
counting.

{\BibTeX} does not work by magic. It doesn't get the bibliographic
data from thin air but from .bib files. If you use {\BibTeX} to produce a
bibliography you must send the .bib files. 

{\LaTeX} can't read your mind. If you assign the same label to a
subsubsection and a table, you might find that Table I has been cross
referenced as Table IV-B3. 

{\LaTeX} does not have precognitive abilities. If you put a
\verb|\label| command before the command that updates the counter it's
supposed to be using, the label will pick up the last counter to be
cross referenced instead. In particular, a \verb|\label| command
should not go before the caption of a figure or a table.

Do not use \verb|\nonumber| inside the \verb|{array}| environment. It
will not stop equation numbers inside \verb|{array}| (there won't be
any anyway) and it might stop a wanted equation number in the
surrounding equation.

\subsection{Some Common Mistakes}\label{SCM}
\begin{itemize}
\item The word ``data'' is plural, not singular.
\item The subscript for the permeability of vacuum $\mu_{0}$, and other common scientific constants, is zero with subscript formatting, not a lowercase letter ``o''.
\item In American English, commas, semicolons, periods, question and exclamation marks are located within quotation marks only when a complete thought or name is cited, such as a title or full quotation. When quotation marks are used, instead of a bold or italic typeface, to highlight a word or phrase, punctuation should appear outside of the quotation marks. A parenthetical phrase or statement at the end of a sentence is punctuated outside of the closing parenthesis (like this). (A parenthetical sentence is punctuated within the parentheses.)
\item A graph within a graph is an ``inset'', not an ``insert''. The word alternatively is preferred to the word ``alternately'' (unless you really mean something that alternates).
\item Do not use the word ``essentially'' to mean ``approximately'' or ``effectively''.
\item In your paper title, if the words ``that uses'' can accurately replace the word ``using'', capitalize the ``u''; if not, keep using lower-cased.
\item Be aware of the different meanings of the homophones ``affect'' and ``effect'', ``complement'' and ``compliment'', ``discreet'' and ``discrete'', ``principal'' and ``principle''.
\item Do not confuse ``imply'' and ``infer''.
\item The prefix ``non'' is not a word; it should be joined to the word it modifies, usually without a hyphen.
\item There is no period after the ``et'' in the Latin abbreviation ``et al.''.
\item The abbreviation ``i.e.'' means ``that is'', and the abbreviation ``e.g.'' means ``for example''.
\end{itemize}
An excellent style manual for science writers is \cite{b7}.

\subsection{Authors and Affiliations}\label{AAA}
\textbf{The class file is designed for, but not limited to, six authors.} A 
minimum of one author is required for all conference articles. Author names 
should be listed starting from left to right and then moving down to the 
next line. This is the author sequence that will be used in future citations 
and by indexing services. Names should not be listed in columns nor group by 
affiliation. Please keep your affiliations as succinct as possible (for 
example, do not differentiate among departments of the same organization).

\subsection{Identify the Headings}\label{ITH}
Headings, or heads, are organizational devices that guide the reader through 
your paper. There are two types: component heads and text heads.

Component heads identify the different components of your paper and are not 
topically subordinate to each other. Examples include Acknowledgments and 
References and, for these, the correct style to use is ``Heading 5''. Use 
``figure caption'' for your Figure captions, and ``table head'' for your 
table title. Run-in heads, such as ``Abstract'', will require you to apply a 
style (in this case, italic) in addition to the style provided by the drop 
down menu to differentiate the head from the text.

Text heads organize the topics on a relational, hierarchical basis. For 
example, the paper title is the primary text head because all subsequent 
material relates and elaborates on this one topic. If there are two or more 
sub-topics, the next level head (uppercase Roman numerals) should be used 
and, conversely, if there are not at least two sub-topics, then no subheads 
should be introduced.

\subsection{Figures and Tables}\label{FAT}
\paragraph{Positioning Figures and Tables} Place figures and tables at the top and 
bottom of columns. Avoid placing them in the middle of columns. Large 
figures and tables may span across both columns. Figure captions should be 
below the figures; table heads should appear above the tables. Insert 
figures and tables after they are cited in the text. Use the abbreviation 
``Fig.~\ref{fig}'', even at the beginning of a sentence.

\begin{table}[htbp]
\caption{Table Type Styles}
\begin{center}
\begin{tabular}{|c|c|c|c|}
\hline
\textbf{Table}&\multicolumn{3}{|c|}{\textbf{Table Column Head}} \\
\cline{2-4} 
\textbf{Head} & \textbf{\textit{Table column subhead}}& \textbf{\textit{Subhead}}& \textbf{\textit{Subhead}} \\
\hline
copy& More table copy$^{\mathrm{a}}$& &  \\
\hline
\multicolumn{4}{l}{$^{\mathrm{a}}$Sample of a Table footnote.}
\end{tabular}
\label{tab1}
\end{center}
\end{table}

\begin{figure}[htbp]
\centerline{\includegraphics{9C Vs 10-Q Vs NCUA Grayscale.png}}
\caption{Filer count FR Y-9C, SEC 10-Q, NCUA Call Reports}
\label{fig}
\end{figure}

Figure Labels: Use 8 point Times New Roman for Figure labels. Use words 
rather than symbols or abbreviations when writing Figure axis labels to 
avoid confusing the reader. As an example, write the quantity 
``Magnetization'', or ``Magnetization, M'', not just ``M''. If including 
units in the label, present them within parentheses. Do not label axes only 
with units. In the example, write ``Magnetization (A/m)'' or ``Magnetization 
\{A[m(1)]\}'', not just ``A/m''. Do not label axes with a ratio of 
quantities and units. For example, write ``Temperature (K)'', not 
``Temperature/K''.

\section*{Acknowledgment}

The preferred spelling of the word ``acknowledgment'' in America is without 
an ``e'' after the ``g''. Avoid the stilted expression ``one of us (R. B. 
G.) thanks $\ldots$''. Instead, try ``R. B. G. thanks$\ldots$''. Put sponsor 
acknowledgments in the unnumbered footnote on the first page.

\section*{References}

Please number citations consecutively within brackets \cite{b1}. The 
sentence punctuation follows the bracket \cite{b2}. Refer simply to the reference 
number, as in \cite{b3}---do not use ``Ref. \cite{b3}'' or ``reference \cite{b3}'' except at 
the beginning of a sentence: ``Reference \cite{b3} was the first $\ldots$''

Number footnotes separately in superscripts. Place the actual footnote at 
the bottom of the column in which it was cited. Do not put footnotes in the 
abstract or reference list. Use letters for table footnotes.

Unless there are six authors or more give all authors' names; do not use 
``et al.''. Papers that have not been published, even if they have been 
submitted for publication, should be cited as ``unpublished'' \cite{b4}. Papers 
that have been accepted for publication should be cited as ``in press'' \cite{b5}. 
Capitalize only the first word in a paper title, except for proper nouns and 
element symbols.

For papers published in translation journals, please give the English 
citation first, followed by the original foreign-language citation \cite{b6}.


\begin{thebibliography}{00}
\bibitem{Cong1} “Bank Holding Companies: Background and Issues for Congress”, [Online]. Available: \url{https://www.everycrsreport.com/reports/R48291.html#:~:text=Contrary%20to%20traditional%20notions%20that,be%20allowed%20to%20engage%20in.}
\bibitem{Fed9C} “Federal Reserve Board - Reporting Forms.” Accessed: Sept. 06, 2025. [Online]. Available: \url{https://www.federalreserve.gov/apps/reportingforms/Report/Index/FR_Y-9C}
\bibitem{SEC10Q} ``SEC Form 10-Q Instructions''[Online] Available: \url{https://www.sec.gov/files/form10-q.pdf} 
\bibitem{MDRM} “The Fed - Micro Data Reference Manual.” Accessed: Sept. 06, 2025. [Online]. Available: https://www.federalreserve.gov/apps/mdrm/
\bibitem{NCUA} [1] “Credit Union and Corporate Call Report Data | NCUA.” Accessed: Sept. 06, 2025. [Online]. Available: https://ncua.gov/analysis/credit-union-corporate-call-report-data
\bibitem{9CCount} ``Fed Register FRY-9C" [Online] Available : \url{https://public-inspection.federalregister.gov/2024-26707.pdf?1731591951}
\bibitem{NCUACount} “Credit Union Assets, Loans Outstanding and Net Income Increase | NCUA.” Accessed: Sept. 06, 2025. [Online]. Available:\url{ https://ncua.gov/newsroom/press-release/2025/credit-union-assets-loans-outstanding-and-net-income-increase?utm_source=chatgpt.com}
\bibitem{SECCount} “Number of Listed Companies for United States (DDOM01USA644NWDB) | FRED | St. Louis Fed.” Accessed: Sept. 06, 2025. [Online]. Available: https://fred.stlouisfed.org/series/DDOM01USA644NWDB
\bibitem{SecRef} Xiangchao Hao, Qinru Sun, Fang Xie, International evidence for the substitution effect of FX derivatives usage on bank capital buffer, Research in International Business and Finance, 10.1016/j.ribaf.2022.101687, 62, (101687), (2022).
\bibitem{der9C} A. R. Abdel-khalik and P.-C. Chen, “Growth in financial derivatives: The public policy and accounting incentives,” Journal of Accounting and Public Policy, vol. 34, no. 3, pp. 291–318, 2015, doi: https://doi.org/10.1016/j.jaccpubpol.2015.01.002.
\bibitem{ASC820} “Fair Value Measurement (Topic 820): Fair Value Measurement of Equity Securities Subject to Contractual Sale Restrictions (Completed Project Summary).” Accessed: Sept. 18, 2025. [Online]. Available: \url {https://fasb.org/page/PageContent?pageId=/projects/recentlycompleted/fair-value-measurement-topic-820-fair-value-measurement-of-equity-securities-subject-to-contractual-sale-restrictions.html}
\bibitem{ASC210} “UPDATE NO. 2013-01—BALANCE SHEET (TOPIC 210): CLARIFYING THE SCOPE OF DISCLOSURES ABOUT OFFSETTING ASSETS AND LIABILITIES.” Accessed: Sept. 18, 2025. [Online]. Available: \url{https://www.fasb.org/page/document?pdf=ASU2013-01.pdf&title=UPDATE%20NO.%202013-01%E2%80%94BALANCE%20SHEET%20(TOPIC%20210):%20CLARIFYING%20THE%20SCOPE%20OF%20DISCLOSURES%20ABOUT%20OFFSETTING%20ASSETS%20AND%20LIABILITIES}
\bibitem{Fed1}[2] J. Berrospide, F. Cai, S. Lewis-Hayre, and F. Zikes, “Bank Lending to Private Credit: Size, Characteristics, and Financial Stability Implications,” May 2025, Accessed: Aug. 18, 2025. [Online]. Available: https://www.federalreserve.gov/econres/notes/feds-notes/bank-lending-to-private-credit-size-characteristics-and-financial-stability-implications-20250523.html
\bibitem{EAR} A. (Aloke) Ghosh, H. Jarva, and S. G. Ryan, “Bank Regulation/Supervision and Bank Auditing,” European Accounting Review, pp. 1–26, Aug. 2024, doi: 10.1080/09638180.2024.2373207.
\bibitem{NII} M. K. Brunnermeier, G. N. Dong, and D. Palia, “Banks’ Noninterest Income and Systemic Risk,” The Review of Corporate Finance Studies, p. cfaa006, May 2020, doi: 10.1093/rcfs/cfaa006.
\bibitem{Fed2} “FDIC Quarterly - Bank and Nonbank Lending over the past 70 years,” Numb er, vol. 1, 2019.
\bibitem{PolkB3}[1] D. Polk and W. Llp, “U.S. Basel III Final Rule: Visual Memorandum”.
\bibitem{ASC810} “Accounting Standards Updates - 2009.” Accessed: Oct. 01, 2025. [Online]. Available: https://asc.fasb.org/1943274/1852705 .
\bibitem{FDICRisk} “FDIC: Search | FDIC.gov.” Accessed: Oct. 06, 2025. [Online]. Available: \url{https://fdic.gov/fdic-search?query=icerc}
\bibitem{FRY11} “Federal Reserve Board - Reporting Forms - FR Y-11/FR Y-11S.” Accessed: Oct. 08, 2025. [Online]. Available: \url{https://www.federalreserve.gov/apps/reportingforms/Report/Index/FR_Y-11FR_Y-11S}
\bibitem{FRY15} “Federal Reserve Board - Reporting Forms - FR Y-15.” Accessed: Oct. 08, 2025. [Online]. Available: \url{https://www.federalreserve.gov/apps/reportingforms/Report/Index/FR_Y-15}
\bibitem{FRY9LP} “Federal Reserve Board - Reporting Forms -FR Y-9LP.” Accessed: Oct. 08, 2025. [Online]. Available: \url{https://www.federalreserve.gov/apps/reportingforms/Report/Index/FR_Y-9LP}
\bibitem{FFIEC002} “FFIEC 002 Current Information | FFIEC - FFIEC-002.” Accessed: Oct. 08, 2025. [Online]. Available: \url{https://www.ffiec.gov/resources/reporting-forms/ffiec002}
\bibitem{FFIEC031} “FFIEC 031 Current Information | FFIEC - FFIEC-031.” Accessed: Oct. 08, 2025. [Online]. Available: \url{https://www.ffiec.gov/resources/reporting-forms/ffiec031}
\bibitem{FFIEC041}“FFIEC 041 Current Information | FFIEC - FFIEC-041.” Accessed: Oct. 08, 2025. [Online]. Available: \url{https://www.ffiec.gov/resources/reporting-forms/ffiec041}





\end{thebibliography}

\section{Appendix}

\subsection{Appendix I - Line descriptions in HC-C for Loan relevant lines}
\begin{table*}[htbp]
	\centering
	\caption{HC-C Loan Categories and Descriptions}
	\begin{tabular}{|p{3cm}|p{12cm}|}
		\hline
		\textbf{Line} & \textbf{Description} \\
		\hline
		1 & Loans secured by real estate. \\
		\hline
		1.a.1 & Construction, land development, and other land loans: 1-4 family residential construction loans. \\
		\hline
		1.a.2 & Construction, land development, and other land loans: Other construction loans and all land development and other land loans. \\
		\hline
		1.b & Secured by farmland. \\
		\hline
		1.c.1 & Secured by 1–4 family residential properties: Revolving, open-end loans secured by 1–4 family residential properties and extended under lines of credit. \\
		\hline
		1.c.2.a & Secured by 1–4 family residential properties: Closed-end loans secured by 1–4 family residential properties — Secured by first liens. \\
		\hline
		1.c.2.b & Secured by 1–4 family residential properties: Closed-end loans secured by 1–4 family residential properties — Secured by junior liens. \\
		\hline
		1.d & Secured by multifamily (5 or more) residential properties. \\
		\hline
		1.e.1 & Secured by nonfarm nonresidential properties — Loans secured by owner-occupied nonfarm nonresidential properties. \\
		\hline
		1.e.2 & Secured by nonfarm nonresidential properties — Loans secured by other nonfarm nonresidential properties. \\
		\hline
		2.a & Loans to depository institutions and acceptances of other banks — To U.S. banks and other U.S. depository institutions. \\
		\hline
		2.b & Loans to depository institutions and acceptances of other banks — To foreign banks. \\
		\hline
		3 Col A & Loans to finance agricultural production and other loans to farmers — Consolidated level. \\
		\hline
		3 Col B & Loans to finance agricultural production and other loans to farmers — Domestic level. \\
		\hline
		4.a & Commercial and Industrial loans — To U.S. addresses (domicile). \\
		\hline
		4.b & Commercial and Industrial loans — To non-U.S. addresses (domicile). \\
		\hline
		4.c & Commercial and Industrial loans — To U.S. addressees and non - US Addresses (domicile). \\
		\hline
		6.a & Loans to individuals for household, family, and other personal expenditures (consumer loans) (includes purchased paper) — Credit cards. \\
		\hline
		6.b & Loans to individuals for household, family, and other personal expenditures (consumer loans) (includes purchased paper) — Other revolving credit plans. \\
		\hline
		6.c & Loans to individuals for household, family, and other personal expenditures (consumer loans) (includes purchased paper) — Automobile loans. \\
		\hline
		6.d & Loans to individuals for household, family, and other personal expenditures (consumer loans) (includes purchased paper) — Other consumer loans (includes single payment, installment, and student loans). \\
		\hline
		7 Col A & Loans to foreign governments and official institutions (including foreign central banks) — Consolidated level. \\
		\hline
		7 Col B & Loans to foreign governments and official institutions (including foreign central banks) — Domestic office level. \\
		\hline
		9.a Col A & Loans to nondepository financial institutions and other loans: Loans to nondepository financial institutions — Consolidated. \\
		\hline
		9.a Col B & Loans to nondepository financial institutions and other loans: Loans to nondepository financial institutions — Domestic offices. \\
		\hline
		9.b.1 Col A & Loans to nondepository financial institutions and other loans: Other loans — Loans for purchasing and carrying securities (secured or unsecured) — Consolidated. \\
		\hline
		9.b.1 Col B & Loans to nondepository financial institutions and other loans: Other loans — Loans for purchasing and carrying securities (secured or unsecured) — Domestic offices. \\
		\hline
		9.b.2 Col A & Loans to nondepository financial institutions and other loans: All other loans (exclude consumer loans) — Consolidated. \\
		\hline
		9.b.2 Col B & Loans to nondepository financial institutions and other loans: All other loans (exclude consumer loans) — Domestic offices. \\
		\hline
		9.b.3 Col A & Loans to nondepository financial institutions and other loans: Loans for purchasing or carrying securities (secured and unsecured) and all other loans — Consolidated. \\
		\hline
		9.b.3 Col B & Loans to nondepository financial institutions and other loans: Loans for purchasing or carrying securities (secured and unsecured) and all other loans — Domestic offices. \\
		\hline
	\end{tabular}
\end{table*}

\subsection{Line descriptions for Loan relevant lines in HC-L}
\begin{table*}[htbp]
	\centering
	\caption{Schedule HC-L Selected Line Items (Unused Commitments, Standby Letters, Credit Derivatives)}
	\begin{tabular}{|p{3cm}|p{12cm}|}
		\hline
		\textbf{Line} & \textbf{Description} \\
		\hline
		1.a & Unused commitments: Revolving, open-end loans secured by 1--4 family residential properties (e.g., home equity lines) \\
		\hline
		1.b.(1) & Unused commitments: Consumer credit card lines \\
		\hline
		1.b.(2) & Unused commitments: Other unused credit card lines \\
		\hline
		1.c.(1) & Unused commitments: Commitments to fund commercial real estate, construction, and land development loans secured by real estate (sum of items 1.c.(1)(a) and (b) must equal item 1.c.(1)) \\
		\hline
		1.c.(1)(a) & Unused commitments: 1--4 family residential construction loan commitments \\
		\hline
		1.c.(1)(b) & Unused commitments: Commercial real estate, other construction loan, and land development loan commitments \\
		\hline
		1.c.(2) & Unused commitments: Commitments to fund commercial real estate, construction, and land development loans NOT secured by real estate \\
		\hline
		1.d & Unused commitments: Securities underwriting \\
		\hline
		1.e.(1) & Unused commitments: Commercial and industrial loans \\
		\hline
		1.e.(2) & Unused commitments: Loans to financial institutions \\
		\hline
		1.e.(3) & Unused commitments: All other unused commitments \\
		\hline
		2.a & Financial standby letters of credit and foreign office guarantees. Amount of financial standby letters of credit conveyed to others (for BHCs with \$5 billion+ in total assets) \\
		\hline
		3 & Performance standby letters of credit and foreign office guarantees \\
		\hline
		3.a & Amount of performance standby letters of credit conveyed to others (for BHCs with \$5 billion+ in total assets) \\
		\hline
		4 & Commercial and similar letters of credit \\
		\hline
		7 & Credit derivatives \\
		\hline
		7.a.(1) Col A & Credit derivatives: Notional amounts --- Credit default swaps (Sold Protection) \\
		\hline
		7.a.(1) Col B & Credit derivatives: Notional amounts --- Credit default swaps (Purchased Protection) \\
		\hline
		7.a.(2) Col A & Credit derivatives: Notional amounts --- Total return swaps (Sold Protection) \\
		\hline
		7.a.(2) Col B & Credit derivatives: Notional amounts --- Total return swaps (Purchased Protection) \\
		\hline
		7.a.(3) Col A & Credit derivatives: Notional amounts --- Credit options (Sold Protection) \\
		\hline
		7.a.(3) Col B & Credit derivatives: Notional amounts --- Credit options (Purchased Protection) \\
		\hline
		7.a.(4) Col A & Credit derivatives: Notional amounts --- Other credit derivatives (Sold Protection) \\
		\hline
		7.a.(4) Col B & Credit derivatives: Notional amounts --- Other credit derivatives (Purchased Protection) \\
		\hline
		7.b.(1) Col A & Credit derivatives: Gross fair values --- Gross positive fair value (Sold Protection) \\
		\hline
		7.b.(1) Col B & Credit derivatives: Gross fair values --- Gross positive fair value (Purchased Protection) \\
		\hline
		7.b.(2) Col A & Credit derivatives: Gross fair values --- Gross negative fair value (Sold Protection) \\
		\hline
		7.b.(2) Col B & Credit derivatives: Gross fair values --- Gross negative fair value (Purchased Protection) \\
		\hline
		7.c.(1)(a) Col A & Credit derivatives: Notional amounts by regulatory capital treatment --- Market Risk Rule: Sold protection (Sold Protection) \\
		\hline
		7.c.(1)(a) Col B & Credit derivatives: Notional amounts by regulatory capital treatment --- Market Risk Rule: Sold protection (Purchased Protection) \\
		\hline
		7.c.(1)(b) Col A & Credit derivatives: Notional amounts by regulatory capital treatment --- Market Risk Rule: Purchased protection (Sold Protection) \\
		\hline
		7.c.(1)(b) Col B & Credit derivatives: Notional amounts by regulatory capital treatment --- Market Risk Rule: Purchased protection (Purchased Protection) \\
		\hline
		7.c.(2)(a) Col A & Credit derivatives: Notional amounts by regulatory capital treatment --- All other positions: Sold protection (Sold Protection) \\
		\hline
		7.c.(2)(a) Col B & Credit derivatives: Notional amounts by regulatory capital treatment --- All other positions: Sold protection (Purchased Protection) \\
		\hline
		7.c.(2)(b) Col A & Credit derivatives: Notional amounts by regulatory capital treatment --- All other positions: Purchased protection recognized for capital purposes (Sold Protection) \\
		\hline
		7.c.(2)(b) Col B & Credit derivatives: Notional amounts by regulatory capital treatment --- All other positions: Purchased protection recognized for capital purposes (Purchased Protection) \\
		\hline
		7.c.(2)(c) Col A & Credit derivatives: Notional amounts by regulatory capital treatment --- All other positions: Purchased protection not recognized for capital purposes (Sold Protection) \\
		\hline
		7.c.(2)(c) Col B & Credit derivatives: Notional amounts by regulatory capital treatment --- All other positions: Purchased protection not recognized for capital purposes (Purchased Protection) \\
		\hline
	\end{tabular}
\end{table*}


\begin{table}[htbp]
	\centering
	\caption{FR Y-9C Loan related Schedule Column Definitions}
	\label{tab:9C schedule_columns}
	\begin{tabular}{|p{3cm}|p{12cm}|}
		\hline
		\textbf{Schedule} & \textbf{Column Description} \\
		\hline
		\multicolumn{2}{|c|}{\textbf{HC-C: Loans and Lease Financing Receivables}} \\
		\hline
		HC-C & Consolidated \\
		\hline
		HC-C & Domestic Offices only \\
		\hline
		\multicolumn{2}{|c|}{\textbf{HC-L: Derivatives and Off-Balance-Sheet Items}} \\
		\hline
		HC-L & Sold Protection (Line 7) \\
		\hline
		HC-L & Purchased Protection (Line 7) \\
		\hline
		HC-L & Interest Rate Contracts (Line 11-14) \\
		\hline
		HC-L & FX Contracts (Line 11-14) \\
		\hline
		HC-L & Equity Derivative Contracts (Line 11-14) \\
		\hline
		HC-L & Commodity/Other Contracts (Line 11-14) \\
		\hline
		\multicolumn{2}{|c|}{\textbf{HC-N: Past Due and Nonaccrual Loans}} \\
		\hline
		HC-N & Past Due 30-89 days \\
		\hline
		HC-N & Past Due more than 90 days and still accruing \\
		\hline
		HC-N & Non Accrual \\
		\hline
		\multicolumn{2}{|c|}{\textbf{HI-B Part I: Charge-offs and Recoveries}} \\
		\hline
		HI-B Part I & Charge offs \\
		\hline
		HI-B Part I & Recoveries \\
		\hline
		\multicolumn{2}{|c|}{\textbf{HI-B Part II: Changes in Allowances for Credit Losses}} \\
		\hline
		HI-B Part II & HFI Loans and Leases \\
		\hline
		HI-B Part II & HTM debt securities \\
		\hline
		HI-B Part II & AFS debt securities \\
		\hline
		\multicolumn{2}{|c|}{\textbf{HI-C: Disaggregated Data on Allowances}} \\
		\hline
		HI-C & Amortized Cost \\
		\hline
		HI-C & Allowance Balance \\
		\hline
		\multicolumn{2}{|c|}{\textbf{HC-Q: Fair Value Measurements}} \\
		\hline
		HC-Q & Total Fair Value Reported on HC \\
		\hline
		HC-Q & Amounts netted in Fair value calculation \\
		\hline
		HC-Q & Level 1 Fair Value measurement \\
		\hline
		HC-Q & Level 2 Fair Value measurement \\
		\hline
		HC-Q & Level 3 Fair Value measurement \\
		\hline
		\multicolumn{2}{|c|}{\textbf{HC-R Part II: Risk-Weighted Assets}} \\
		\hline
		HC-R Part II & Total from HC \\
		\hline
		HC-R Part II & Adjustments to totals reported in column A \\
		\hline
		HC-R Part II & 0\% \\
		\hline
		HC-R Part II & 2\% \\
		\hline
		HC-R Part II & 4\% \\
		\hline
		HC-R Part II & 10\% \\
		\hline
		HC-R Part II & 20\% \\
		\hline
		HC-R Part II & 50\% \\
		\hline
		HC-R Part II & 100\% \\
		\hline
		HC-R Part II & 150\% \\
		\hline
		HC-R Part II & 250\% \\
		\hline
		HC-R Part II & 300\% \\
		\hline
		HC-R Part II & 400\% \\
		\hline
		HC-R Part II & 600\% \\
		\hline
		HC-R Part II & 625\% \\
		\hline
		HC-R Part II & 937.50\% \\
		\hline
		HC-R Part II & 1250\% \\
		\hline
		HC-R Part II & Exposure Amount \\
		\hline
		HC-R Part II & RWA Amount \\
		\hline
		HC-R Part II & SSFA Amount (Line 9,10) \\
		\hline
		HC-R Part II & Gross Up Amount (Line 9,10) \\
		\hline
		HC-R Part II & Credit Equivalent Amount (for off balance sheet lines) \\
		\hline
		\multicolumn{2}{|c|}{\textbf{HC-S: Servicing, Securitization, and Asset Sale Activities}} \\
		\hline
		HC-S & 1-4 Family residential loans \\
		\hline
		HC-S & Home Equity loans \\
		\hline
		HC-S & Credit Card Loans \\
		\hline
		HC-S & Auto Loans \\
		\hline
		HC-S & Other consumer Loans \\
		\hline
		HC-S & Commercial and Industrial Loans \\
		\hline
		HC-S & All Other Loans \\
		\hline
		\multicolumn{2}{|c|}{\textbf{HC-V: Variable Interest Entities}} \\
		\hline
		HC-V & Securitization Vehicles \\
		\hline
		HC-V & Other VIEs \\
		\hline
	\end{tabular}
\end{table}


\vspace{12pt}
\color{red}
IEEE conference templates contain guidance text for composing and formatting conference papers. Please ensure that all template text is removed from your conference paper prior to submission to the conference. Failure to remove the template text from your paper may result in your paper not being published.


\end{document}
